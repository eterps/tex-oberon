\input taocpmac

\beginchapter CHAPTER 1: HISTORY AND MOTIVATION.

{\quoteformat{The most important thing in the programming language is the name. A language will not succeed without a good name. I have recently invented a very good name and now I am looking for a suitable language.}
\author Donald Knuth (19??)
}

\medskip

How could anyone diligently concentrate on his work on an afternoon
with such warmth, splendid sunshine, and blue sky. This rhetorical
question was one I asked many times while spending a sabbatical leave
in California in 1985. Back home everyone would feel compelled to
profit from the sunny spells to enjoy life at leisure in the
country-side, wandering or engaging in one's favourite sport. But
here, every day was like that, and giving in to such temptations would
have meant the end of all work. And, had I not chosen this location in
the world because of its inviting, enjoyable climate?

Fortunately, my work was also enticing, making it easier to buckle
down. I had the privilege of sitting in front of the most advanced and
powerful workstation anywhere, learning the secrets of perhaps the
newest fad in our fast developing trade, pushing colored rectangles
from one place of the screen to another. This all had to happen under
strict observance of rules imposed by physical laws and by the newest
technology. Fortunately, the advanced computer would complain
immediately if such a rule was violated, it was a rule checker and
acted like your big brother, preventing you from making steps towards
disaster. And it did what would have been impossible for oneself,
keeping track of thousands of constraints among the thousands of
rectangles laid out. This was called computer-aided design. ``Aided'' is
rather a euphemism, but the computer did not complain about the
degradation of its role.

While my eyes were glued to the colorful display, and while I was
confronted with the evidence of my latest inadequacy, in through the
always open door stepped my colleague (JG). He also happened to spend
a leave from duties at home at the same laboratory, yet his face did
not exactly express happiness, but rather frustration. The chocolate
bar in his hand did for him what the coffee cup or the pipe does for
others, providing temporary relaxation and distraction. It was not the
first time he appeared in this mood, and without words I guessed its
cause. And the episode would reoccur many times.

His days were not filled with the great fun of rectangle-pushing; he
had an assignment. He was charged with the design of a compiler for
the same advanced computer. Therefore, he was forced to deal much more
closely, if not intimately, with the underlying software system. Its
rather frequent failures had to be understood in his case, for he was
programming, whereas I was only using it through an application; in
short, I was an end-user! These failures had to be understood not for
purposes of correction, but in order to find ways to avoid them. How
was the necessary insight to be obtained? I realized at this moment
that I had so far avoided this question; I had limited familiarization
with this novel system to the bare necessities which sufficed for the
task on my mind.

It soon became clear that a study of the system was nearly
impossible. Its dimensions were simply awesome, and documentation
accordingly sparse. Answers to questions that were momentarily
pressing could best be obtained by interviewing the system's
designers, who all were in-house. In doing so, we made the shocking
discovery that often we could not understand their
language. Explanations were fraught with jargon and references to
other parts of the system which had remained equally enigmatic to us.

So, our frustration-triggered breaks from compiler construction and
chip design became devoted to attempts to identify the essence, the
foundations of the system's novel aspects. What made it different from
conventional operating systems? Which of these concepts were
essential, which ones could be improved, simplified, or even
discarded? And where were they rooted? Could the system's essence be
distilled and extracted, like in a chemical process?

During the ensuing discussions, the idea emerged slowly to undertake
our own design. And suddenly it had become concrete. ``crazy'' was my
first reaction, and ``impossible''. The sheer amount of work appeared as
overwhelming. After all, we both had to carry our share of teaching
duties back home. But the thought was implanted and continued to
occupy our minds.

Sometime thereafter, events back home suggested that I should take
over the important course about System Software. As it was the
unwritten rule that it should primarily deal with operating system
principles, I hesitated. My scruples were easily justified: After all
I had never designed such a system nor a part of it. And how can one
teach an engineering subject without first-hand experience?

Impossible? Had we not designed compilers, operating systems, and
document editors in small teams? And had I not repeatedly experienced
that an inadequate and frustrating program could be programmed from
scratch in a fraction of source code used by the original design? Our
brain-storming continued, with many intermissions, over several
weeks, and certain shapes of a system structure slowly emerged through
the haze. After some time, the preposterous decision was made: we
would embark on the design of an operating system for our workstation
(which happened to be much less powerful than the one used for my
rectangle-pushing) from scratch.

The primary goal, to personally obtain first-hand experience, and to
reach full understanding of every detail, inherently determined our
manpower: two part-time programmers. We tentatively set our time-limit
for completion to three years. As it later turned out, this had been a
good estimate; programming was begun in early 1986, and a first
version of the system was released in the fall of 1988.

Although the search for an appropriate name for a project is usually a
minor problem and often left to chance and whim of the designers, this
may be the place to recount how Oberon entered the picture in our
case. It happened that around the time of the beginning of our effort,
the space probe Voyager made headlines with a series of spectacular
pictures taken of the planet Uranus and of its moons, the largest of
which is named Oberon. Since its launch I had considered the Voyager
project as a singularly well-planned and successful endeavor, and as a
small tribute to it I picked the name of its latest object of
investigation. There are indeed very few engineering projects whose
products perform way beyond expectations and beyond their anticipated
lifetime; mostly they fail much earlier, particularly in the domain of
software. And, last but not least, we recall that Oberon is famous as
the king of elfs.

The consciously planned shortage of manpower enforced a single, but
healthy, guideline: Concentrate on essential functions and omit
embellishments that merely cater to established conventions and
passing tastes. Of course, the essential core had first to be
recognized and crystallized. But the basis had been laid. The ground
rule became even more crucial when we decided that the result should
be able to be used as teaching material. I remembered C.A.R. Hoare's
plea that books should be written presenting actually operational
systems rather than half-baked, abstract principles. He had
complained in the early 1970s that in our field engineers were told to
constantly create new artifacts without being given the chance to
study previous works that had proven their worth in the field. How
right was he, even to the present day!

The emerging goal to publish the result with all its details let the
choice of programming language appear in a new light: it became
crucial. Modula-2 which we had planned to use, appeared as not quite
satisfactory. Firstly, it lacked a facility to express extensibility
in an adequate way. And we had put extensibility among the principal
properties of the new system. By ``adequate'' we include
machine-independence. Our programs should be expressed in a manner
that makes no reference to machine peculiarities and low-level
programming facilities, perhaps with the exception of device
interfaces, where dependence is inherent.

Hence, Modula-2 was extended with a feature that is now known as type
extension. We also recognized that Modula-2 contained several
facilities that we would not need, that do not genuinely contribute to
its power of expression, but at the same time increase the complexity
of the compiler. But the compiler would not only have to be
implemented, but also to be described, studied, and understood. This
led to the decision to start from a clean slate also in the domain of
language design, and to apply the same principle to it: concentrate on
the essential, purge the rest. The new language, which still bears
much resemblance to Modula-2, was given the same name as the system:
Oberon
\footnote1{N. Wirth. The programming language Oberon. Software - Practice and Experience 18, 7, (July 1988) 671-690.}
\footnote2{M. Reiser and N. Wirth. Programming in Oberon - Steps beyond Pascal and Modula. Addison- Wesley, 1992.}
In contrast to its ancestor it is terser and, above
all, a significant step towards expressing programs on a high level of
abstraction without reference to machine-specific features.

We started designing the system in late fall 1985, and programming in
early 1986. As a vehicle we used our workstation Lilith and its
language Modula-2. First, a cross-compiler was developed, then
followed the modules of the inner core together with the necessary
testing and down-loading facilities. The development of the display
and the text system proceeded simultaneously, without the possibility
of testing, of course. We learned how the absence of a debugger, and
even more so the absence of a compiler, can contribute to careful
programming.

Thereafter followed the translation of the compiler into Oberon. This
was swiftly done, because the original had been written with
anticipation of the later translation. After its availability on the
target computer Ceres, together with the operability of the text
editing facility, the umbilical cord to Lilith could be cut off. The
Oberon System had become real, at least its draft version. This
happened around the middle of 1987; its description was published
thereafter
\footnote3{N. Wirth and J. Gutknecht. The Oberon System. Software - Practice and Experience, 19, 9 (Sept. 1989), 857-893.}, and a manual and guide followed in 1991
\footnote5{M. Reiser. The Oberon System - User Guide and Programmer's Manual. Addison-Wesley, 1991.}.

The system's completion took another year and concentrated on
connecting the workstations in a network for file transfer
\footnote4{N. Wirth. Ceres-Net: A low-cost computer network. Software - Practice and Experience, 20, 1 (Jan. 1990), 13-24.}, on a
central printing facility, and on maintenance tools. The goal of
completing the system within three years had been met. The system was
introduced in the middle of 1988 to a wider user community, and work
on applications could start. A service for electronic mail was
developed, a graphics system was added, and various efforts for
general document preparation systems proceeded. The display facility
was extended to accommodate two screens, including color. At the same
time, feedback from experience in its use was incorporated by
improving existing parts. Since 1989, Oberon has replaced Modula-2 in
our introductory programming courses.

\beginchapter CHAPTER 2: STRUCTURE OF THE SYSTEM.

\medskip

\beginsubsection 2.1. INTRODUCTION.

In order to warrant the sizeable effort of designing and constructing
an entire operating system from scratch, a number of basic concepts
need to be novel. We start this chapter with a discussion of the
principal concepts underlying the Oberon System and of the dominant
design decisions. On this basis, a presentation of the system's
structure follows. It will be restricted to its coarsest level, namely
the composition and interdependence of the largest building blocks,
the modules. The chapter ends with an overview of the remainder of the
book. It should help the reader to understand the role, place, and
significance of the parts described in the individual chapters.

The fundamental objective of an operating system is to present the
computer to the user and to the programmer at a certain level of
abstraction. For example, the store is presented in terms of
requestable pieces or variables of a specified data type, the disk is
presented in terms of sequences of characters (or bytes) called files,
the display is presented as rectangular areas called viewers, the
keyboard is presented as an input stream of characters, and the mouse
appears as a pair of coordinates and a set of key states. Every
abstraction is characterized by certain properties and governed by a
set of operations. It is the task of the system to implement these
operations and to manage them, constrained by the available resources
of the underlying computer. This is commonly called resource
management.

Every abstraction inherently hides details, namely those from which it
abstracts. Hiding may occur at different levels. For example, the
computer may allow certain parts of the store, or certain devices to
be made inaccessible according to its mode of operation
(user/supervisor mode), or the programming language may make certain
parts inaccessible through a hiding facility inherent in its
visibility rules. The latter is of course much more flexible and
powerful, and the former indeed plays an almost negligible role in our
system. Hiding is important because it allows maintenance of certain
properties (called invariants) of an abstraction to be
guaranteed. Abstraction is indeed the key of any modularization, and
without modularization every hope of being able to guarantee
reliability and correctness vanishes. Clearly, the Oberon System was
designed with the goal of establishing a modular structure on the
basis of purpose-oriented abstractions. The availability of an
appropriate programming language is an indispensable prerequisite, and
the importance of its choice cannot be over-emphasized.

\beginsubsection 2.2. CONCEPTS.

\beginsubsubsection 2.2.1. Viewers.

Whereas the abstractions of individual variables representing parts of
the primary store, and of files representing parts of the disk store
are well established notions and have significance in every computer
system, abstractions regarding input and output devices became
important with the advent of high interactivity between user and
computer. High interactivity requires high bandwidth, and the only
channel of human users with high bandwidth is the eye. Consequently,
the computer's visual output unit must be properly matched with the
human eye. This occurred with the advent of the high-resolution
display in the mid 1970s, which in turn had become feasible due to
faster and cheaper electronic memory components. The high-resolution
display marked one of the few very significant break-throughs in the
history of computer development. The typical bandwidth of a modern
display is in the order of 100 MHz. Primarily the high-resolution
display made visual output a subject of abstraction and resource
management. In the Oberon System, the display is partitioned into
viewers, also called windows, or more precisely, into frames,
rectangular areas of the screen(s). A viewer typically consists of two
frames, a title bar containing a subject name and a menu of commands,
and a main frame containing some text, graphic, picture, or other
object. A viewer itself is a frame; frames can be nested, in principle
to any depth.  14 The System provides routines for generating a frame
(viewer), for moving and for closing it. It allocates a new viewer at
a specified place, and upon request delivers hints as to where it
might best be placed. It keeps track of the set of opened
viewers. This is what is called viewer management, in contrast to the
handling of their displayed contents.  But high interactivity requires
not only a high bandwidth for visual output, it demands also
flexibility of input. Surely, there is no need for an equally large
bandwidth, but a keyboard limited by the speed of typing to about 100
Hz is not good enough. The break-through on this front was achieved by
the so-called mouse, a pointing device which appeared roughly at the
same time as the high- resolution display.  This was by no means just
a lucky coincidence. The mouse comes to fruition only through
appropriate software and the high-resolution display. It is itself a
conceptually very simple device delivering signals when moved on the
table. These signals allow the computer to update the position of a
mark - the cursor - on the display. Since feedback occurs through the
human eye, no great precision is required from the mouse. For example,
when the user wishes to identify a certain object on the screen, such
as a letter, he moves the mouse as long as required until the mapped
cursor reaches the object. This stands in marked contrast to a
digitizer which is supposed to deliver exact coordinates. The Oberon
System relies very much on the availability of a mouse.  Perhaps the
cleverest idea was to equip mice with buttons. By being able to signal
a request with the same hand that determines the cursor position, the
user obtains the direct impression of issuing position-dependent
requests. Position-dependence is realized in software by delegating
interpretation of the signal to a procedure - a so-called handler or
interpreter -which is local to the viewer in whose area the cursor
momentarily appears. A surprising flexibility of command activation
can be achieved in this manner by appropriate software. Various
techniques have emerged in this connection, e.g. pop-up menus,
pull-down-menus, etc. which are powerful even under the presence of a
single button only. For many applications, a mouse with several keys
is far superior, and the Oberon System basically assumes three buttons
to be available. The assignment of different functions to the keys may
of course easily lead to confusion when every application prescribes
different key assignment. This is, however, easily avoided by the
adherence to certain "global" conventions. In the Oberon System, the
left button is primarily used for marking a position (setting a
caret), the middle button for issuing general commands (see below),
and the right button for selecting displayed objects.  Recently, it
has become fashionable to use overlapping windows mirroring documents
being piled up on one's desk. We have found this metaphor not entirely
convincing. Partially hidden windows are typically brought to the top
and made fully visible before any operation is applied to their
contents. In contrast to the insignificant advantage stands the
substantial effort necessary to implement this scheme. It is a good
example of a case where the benefit of a complication is
incommensurate with its cost. Therefore, we have chosen a solution
that is much simpler to realize, yet has no genuine disadvantages
compared to overlapping windows: tiled viewers as shown in Fig. 2.1.
2.2.2. Commands Position-dependent commands with fixed meaning (fixed
for each type of viewer) must be supplemented by general
commands. Conventionally, such commands are issued through the
keyboard by typing the program's name that is to be executed into a
special command text. In this respect, the Oberon System offers a
novel and much more flexible solution which is presented in the
following paragraphs.  First of all we remark that a program in the
common sense of a text compiled as a unit is mostly a far too large
unit of action to serve as a command. Compare it, for example, with
the insertion of a piece of text through a mouse command. In Oberon,
the notion of a unit of action is separated from the notion of unit of
compilation. The former is a command represented by a (exported)
procedure, the latter is a module. Hence, a module may, and typically
does, define several, even many commands. Such a (general) command may
be invoked at any time by pointing at its name in any 15

text visible in any viewer on the display, and by clicking the middle mouse button. The command name has the form M.P, where P is the procedure's identifier and M that of the module in which P is declared. As a consequence, any command click may cause the loading of one or several modules, if M is not already present in main store. The next invocation of M.P occurs instantaneously, since M is already loaded. A further consequence is that modules are never (automatically) removed, because a next command may well refer to the same module.

\bye
