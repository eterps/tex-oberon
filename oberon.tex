\input taocpmac

\beginchapter CHAPTER 1: HISTORY AND MOTIVATION.

{\quoteformat{The most important thing in the programming language is the name. A language will not succeed without a good name. I have recently invented a very good name and now I am looking for a suitable language.}
\author Donald Knuth (19??)
}

\medskip

How could anyone diligently concentrate on his work on an afternoon
with such warmth, splendid sunshine, and blue sky. This rhetorical
question was one I asked many times while spending a sabbatical leave
in California in 1985. Back home everyone would feel compelled to
profit from the sunny spells to enjoy life at leisure in the
country-side, wandering or engaging in one's favourite sport. But
here, every day was like that, and giving in to such temptations would
have meant the end of all work. And, had I not chosen this location in
the world because of its inviting, enjoyable climate?

Fortunately, my work was also enticing, making it easier to buckle
down. I had the privilege of sitting in front of the most advanced and
powerful workstation anywhere, learning the secrets of perhaps the
newest fad in our fast developing trade, pushing colored rectangles
from one place of the screen to another. This all had to happen under
strict observance of rules imposed by physical laws and by the newest
technology. Fortunately, the advanced computer would complain
immediately if such a rule was violated, it was a rule checker and
acted like your big brother, preventing you from making steps towards
disaster. And it did what would have been impossible for oneself,
keeping track of thousands of constraints among the thousands of
rectangles laid out. This was called computer-aided design. ``Aided'' is
rather a euphemism, but the computer did not complain about the
degradation of its role.

While my eyes were glued to the colorful display, and while I was
confronted with the evidence of my latest inadequacy, in through the
always open door stepped my colleague (JG). He also happened to spend
a leave from duties at home at the same laboratory, yet his face did
not exactly express happiness, but rather frustration. The chocolate
bar in his hand did for him what the coffee cup or the pipe does for
others, providing temporary relaxation and distraction. It was not the
first time he appeared in this mood, and without words I guessed its
cause. And the episode would reoccur many times.

His days were not filled with the great fun of rectangle-pushing; he
had an assignment. He was charged with the design of a compiler for
the same advanced computer. Therefore, he was forced to deal much more
closely, if not intimately, with the underlying software system. Its
rather frequent failures had to be understood in his case, for he was
programming, whereas I was only using it through an application; in
short, I was an end-user! These failures had to be understood not for
purposes of correction, but in order to find ways to avoid them. How
was the necessary insight to be obtained? I realized at this moment
that I had so far avoided this question; I had limited familiarization
with this novel system to the bare necessities which sufficed for the
task on my mind.

It soon became clear that a study of the system was nearly
impossible. Its dimensions were simply awesome, and documentation
accordingly sparse. Answers to questions that were momentarily
pressing could best be obtained by interviewing the system's
designers, who all were in-house. In doing so, we made the shocking
discovery that often we could not understand their
language. Explanations were fraught with jargon and references to
other parts of the system which had remained equally enigmatic to us.

So, our frustration-triggered breaks from compiler construction and
chip design became devoted to attempts to identify the essence, the
foundations of the system's novel aspects. What made it different from
conventional operating systems? Which of these concepts were
essential, which ones could be improved, simplified, or even
discarded? And where were they rooted? Could the system's essence be
distilled and extracted, like in a chemical process?

During the ensuing discussions, the idea emerged slowly to undertake
our own design. And suddenly it had become concrete. ``crazy'' was my
first reaction, and ``impossible''. The sheer amount of work appeared as
overwhelming. After all, we both had to carry our share of teaching
duties back home. But the thought was implanted and continued to
occupy our minds.

Sometime thereafter, events back home suggested that I should take
over the important course about System Software. As it was the
unwritten rule that it should primarily deal with operating system
principles, I hesitated. My scruples were easily justified: After all
I had never designed such a system nor a part of it. And how can one
teach an engineering subject without first-hand experience?

Impossible? Had we not designed compilers, operating systems, and
document editors in small teams? And had I not repeatedly experienced
that an inadequate and frustrating program could be programmed from
scratch in a fraction of source code used by the original design? Our
brain-storming continued, with many intermissions, over several
weeks, and certain shapes of a system structure slowly emerged through
the haze. After some time, the preposterous decision was made: we
would embark on the design of an operating system for our workstation
(which happened to be much less powerful than the one used for my
rectangle-pushing) from scratch.

The primary goal, to personally obtain first-hand experience, and to
reach full understanding of every detail, inherently determined our
manpower: two part-time programmers. We tentatively set our time-limit
for completion to three years. As it later turned out, this had been a
good estimate; programming was begun in early 1986, and a first
version of the system was released in the fall of 1988.

Although the search for an appropriate name for a project is usually a
minor problem and often left to chance and whim of the designers, this
may be the place to recount how Oberon entered the picture in our
case. It happened that around the time of the beginning of our effort,
the space probe Voyager made headlines with a series of spectacular
pictures taken of the planet Uranus and of its moons, the largest of
which is named Oberon. Since its launch I had considered the Voyager
project as a singularly well-planned and successful endeavor, and as a
small tribute to it I picked the name of its latest object of
investigation. There are indeed very few engineering projects whose
products perform way beyond expectations and beyond their anticipated
lifetime; mostly they fail much earlier, particularly in the domain of
software. And, last but not least, we recall that Oberon is famous as
the king of elfs.

The consciously planned shortage of manpower enforced a single, but
healthy, guideline: Concentrate on essential functions and omit
embellishments that merely cater to established conventions and
passing tastes. Of course, the essential core had first to be
recognized and crystallized. But the basis had been laid. The ground
rule became even more crucial when we decided that the result should
be able to be used as teaching material. I remembered C.A.R. Hoare's
plea that books should be written presenting actually operational
systems rather than half-baked, abstract principles. He had
complained in the early 1970s that in our field engineers were told to
constantly create new artifacts without being given the chance to
study previous works that had proven their worth in the field. How
right was he, even to the present day!

The emerging goal to publish the result with all its details let the
choice of programming language appear in a new light: it became
crucial. Modula-2 which we had planned to use, appeared as not quite
satisfactory. Firstly, it lacked a facility to express extensibility
in an adequate way. And we had put extensibility among the principal
properties of the new system. By ``adequate'' we include
machine-independence. Our programs should be expressed in a manner
that makes no reference to machine peculiarities and low-level
programming facilities, perhaps with the exception of device
interfaces, where dependence is inherent.

Hence, Modula-2 was extended with a feature that is now known as type
extension. We also recognized that Modula-2 contained several
facilities that we would not need, that do not genuinely contribute to
its power of expression, but at the same time increase the complexity
of the compiler. But the compiler would not only have to be
implemented, but also to be described, studied, and understood. This
led to the decision to start from a clean slate also in the domain of
language design, and to apply the same principle to it: concentrate on
the essential, purge the rest. The new language, which still bears
much resemblance to Modula-2, was given the same name as the system:
Oberon
\footnote1{N. Wirth. The programming language Oberon. Software - Practice and Experience 18, 7, (July 1988) 671-690.}
\footnote2{M. Reiser and N. Wirth. Programming in Oberon - Steps beyond Pascal and Modula. Addison- Wesley, 1992.}
In contrast to its ancestor it is terser and, above
all, a significant step towards expressing programs on a high level of
abstraction without reference to machine-specific features.

We started designing the system in late fall 1985, and programming in
early 1986. As a vehicle we used our workstation Lilith and its
language Modula-2. First, a cross-compiler was developed, then
followed the modules of the inner core together with the necessary
testing and down-loading facilities. The development of the display
and the text system proceeded simultaneously, without the possibility
of testing, of course. We learned how the absence of a debugger, and
even more so the absence of a compiler, can contribute to careful
programming.

Thereafter followed the translation of the compiler into Oberon. This
was swiftly done, because the original had been written with
anticipation of the later translation. After its availability on the
target computer Ceres, together with the operability of the text
editing facility, the umbilical cord to Lilith could be cut off. The
Oberon System had become real, at least its draft version. This
happened around the middle of 1987; its description was published
thereafter
\footnote3{N. Wirth and J. Gutknecht. The Oberon System. Software - Practice and Experience, 19, 9 (Sept. 1989), 857-893.}, and a manual and guide followed in 1991
\footnote5{M. Reiser. The Oberon System - User Guide and Programmer's Manual. Addison-Wesley, 1991.}.

The system's completion took another year and concentrated on
connecting the workstations in a network for file transfer
\footnote4{N. Wirth. Ceres-Net: A low-cost computer network. Software - Practice and Experience, 20, 1 (Jan. 1990), 13-24.}, on a
central printing facility, and on maintenance tools. The goal of
completing the system within three years had been met. The system was
introduced in the middle of 1988 to a wider user community, and work
on applications could start. A service for electronic mail was
developed, a graphics system was added, and various efforts for
general document preparation systems proceeded. The display facility
was extended to accommodate two screens, including color. At the same
time, feedback from experience in its use was incorporated by
improving existing parts. Since 1989, Oberon has replaced Modula-2 in
our introductory programming courses.

\bye
