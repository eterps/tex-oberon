\input taocpmac

\beginchapter CHAPTER 1: HISTORY AND MOTIVATION.

{\quoteformat{The most important thing in the programming language is the name. A language will not succeed without a good name. I have recently invented a very good name and now I am looking for a suitable language.}
\author Donald Knuth (19??)
}

\medskip

How could anyone diligently concentrate on his work on an afternoon
with such warmth, splendid sunshine, and blue sky. This rhetorical
question was one I asked many times while spending a sabbatical leave
in California in 1985. Back home everyone would feel compelled to
profit from the sunny spells to enjoy life at leisure in the
country-side, wandering or engaging in one's favourite sport. But
here, every day was like that, and giving in to such temptations would
have meant the end of all work. And, had I not chosen this location in
the world because of its inviting, enjoyable climate?

Fortunately, my work was also enticing, making it easier to buckle
down. I had the privilege of sitting in front of the most advanced and
powerful workstation anywhere, learning the secrets of perhaps the
newest fad in our fast developing trade, pushing colored rectangles
from one place of the screen to another. This all had to happen under
strict observance of rules imposed by physical laws and by the newest
technology. Fortunately, the advanced computer would complain
immediately if such a rule was violated, it was a rule checker and
acted like your big brother, preventing you from making steps towards
disaster. And it did what would have been impossible for oneself,
keeping track of thousands of constraints among the thousands of
rectangles laid out. This was called computer-aided design. ``Aided'' is
rather a euphemism, but the computer did not complain about the
degradation of its role.

While my eyes were glued to the colorful display, and while I was
confronted with the evidence of my latest inadequacy, in through the
always open door stepped my colleague (JG). He also happened to spend
a leave from duties at home at the same laboratory, yet his face did
not exactly express happiness, but rather frustration. The chocolate
bar in his hand did for him what the coffee cup or the pipe does for
others, providing temporary relaxation and distraction. It was not the
first time he appeared in this mood, and without words I guessed its
cause. And the episode would reoccur many times.

His days were not filled with the great fun of rectangle-pushing; he
had an assignment. He was charged with the design of a compiler for
the same advanced computer. Therefore, he was forced to deal much more
closely, if not intimately, with the underlying software system. Its
rather frequent failures had to be understood in his case, for he was
programming, whereas I was only using it through an application; in
short, I was an end-user! These failures had to be understood not for
purposes of correction, but in order to find ways to avoid them. How
was the necessary insight to be obtained? I realized at this moment
that I had so far avoided this question; I had limited familiarization
with this novel system to the bare necessities which sufficed for the
task on my mind.

It soon became clear that a study of the system was nearly
impossible. Its dimensions were simply awesome, and documentation
accordingly sparse. Answers to questions that were momentarily
pressing could best be obtained by interviewing the system's
designers, who all were in-house. In doing so, we made the shocking
discovery that often we could not understand their
language. Explanations were fraught with jargon and references to
other parts of the system which had remained equally enigmatic to us.

So, our frustration-triggered breaks from compiler construction and
chip design became devoted to attempts to identify the essence, the
foundations of the system's novel aspects. What made it different from
conventional operating systems? Which of these concepts were
essential, which ones could be improved, simplified, or even
discarded? And where were they rooted? Could the system's essence be
distilled and extracted, like in a chemical process?

During the ensuing discussions, the idea emerged slowly to undertake
our own design. And suddenly it had become concrete. ``crazy'' was my
first reaction, and ``impossible''. The sheer amount of work appeared as
overwhelming. After all, we both had to carry our share of teaching
duties back home. But the thought was implanted and continued to
occupy our minds.

Sometime thereafter, events back home suggested that I should take
over the important course about System Software. As it was the
unwritten rule that it should primarily deal with operating system
principles, I hesitated. My scruples were easily justified: After all
I had never designed such a system nor a part of it. And how can one
teach an engineering subject without first-hand experience?

Impossible? Had we not designed compilers, operating systems, and
document editors in small teams? And had I not repeatedly experienced
that an inadequate and frustrating program could be programmed from
scratch in a fraction of source code used by the original design? Our
brain-storming continued, with many intermissions, over several
weeks, and certain shapes of a system structure slowly emerged through
the haze. After some time, the preposterous decision was made: we
would embark on the design of an operating system for our workstation
(which happened to be much less powerful than the one used for my
rectangle-pushing) from scratch.

The primary goal, to personally obtain first-hand experience, and to
reach full understanding of every detail, inherently determined our
manpower: two part-time programmers. We tentatively set our time-limit
for completion to three years. As it later turned out, this had been a
good estimate; programming was begun in early 1986, and a first
version of the system was released in the fall of 1988.

Although the search for an appropriate name for a project is usually a
minor problem and often left to chance and whim of the designers, this
may be the place to recount how Oberon entered the picture in our
case. It happened that around the time of the beginning of our effort,
the space probe Voyager made headlines with a series of spectacular
pictures taken of the planet Uranus and of its moons, the largest of
which is named Oberon. Since its launch I had considered the Voyager
project as a singularly well-planned and successful endeavor, and as a
small tribute to it I picked the name of its latest object of
investigation. There are indeed very few engineering projects whose
products perform way beyond expectations and beyond their anticipated
lifetime; mostly they fail much earlier, particularly in the domain of
software. And, last but not least, we recall that Oberon is famous as
the king of elfs.

The consciously planned shortage of manpower enforced a single, but
healthy, guideline: Concentrate on essential functions and omit
embellishments that merely cater to established conventions and
passing tastes. Of course, the essential core had first to be
recognized and crystallized. But the basis had been laid. The ground
rule became even more crucial when we decided that the result should
be able to be used as teaching material. I remembered C.A.R. Hoare's
plea that books should be written presenting actually operational
systems rather than half-baked, abstract principles. He had
complained in the early 1970s that in our field engineers were told to
constantly create new artifacts without being given the chance to
study previous works that had proven their worth in the field. How
right was he, even to the present day!

The emerging goal to publish the result with all its details let the
choice of programming language appear in a new light: it became
crucial. Modula-2 which we had planned to use, appeared as not quite
satisfactory. Firstly, it lacked a facility to express extensibility
in an adequate way. And we had put extensibility among the principal
properties of the new system. By ``adequate'' we include
machine-independence. Our programs should be expressed in a manner
that makes no reference to machine peculiarities and low-level
programming facilities, perhaps with the exception of device
interfaces, where dependence is inherent.

Hence, Modula-2 was extended with a feature that is now known as type
extension. We also recognized that Modula-2 contained several
facilities that we would not need, that do not genuinely contribute to
its power of expression, but at the same time increase the complexity
of the compiler. But the compiler would not only have to be
implemented, but also to be described, studied, and understood. This
led to the decision to start from a clean slate also in the domain of
language design, and to apply the same principle to it: concentrate on
the essential, purge the rest. The new language, which still bears
much resemblance to Modula-2, was given the same name as the system:
Oberon
\footnote1{N. Wirth. The programming language Oberon. Software - Practice and Experience 18, 7, (July 1988) 671-690.}
\footnote2{M. Reiser and N. Wirth. Programming in Oberon - Steps beyond Pascal and Modula. Addison- Wesley, 1992.}
In contrast to its ancestor it is terser and, above
all, a significant step towards expressing programs on a high level of
abstraction without reference to machine-specific features.

We started designing the system in late fall 1985, and programming in
early 1986. As a vehicle we used our workstation Lilith and its
language Modula-2. First, a cross-compiler was developed, then
followed the modules of the inner core together with the necessary
testing and down-loading facilities. The development of the display
and the text system proceeded simultaneously, without the possibility
of testing, of course. We learned how the absence of a debugger, and
even more so the absence of a compiler, can contribute to careful
programming.

Thereafter followed the translation of the compiler into Oberon. This
was swiftly done, because the original had been written with
anticipation of the later translation. After its availability on the
target computer Ceres, together with the operability of the text
editing facility, the umbilical cord to Lilith could be cut off. The
Oberon System had become real, at least its draft version. This
happened around the middle of 1987; its description was published
thereafter
\footnote3{N. Wirth and J. Gutknecht. The Oberon System. Software - Practice and Experience, 19, 9 (Sept. 1989), 857-893.}, and a manual and guide followed in 1991
\footnote5{M. Reiser. The Oberon System - User Guide and Programmer's Manual. Addison-Wesley, 1991.}.

The system's completion took another year and concentrated on
connecting the workstations in a network for file transfer
\footnote4{N. Wirth. Ceres-Net: A low-cost computer network. Software - Practice and Experience, 20, 1 (Jan. 1990), 13-24.}, on a
central printing facility, and on maintenance tools. The goal of
completing the system within three years had been met. The system was
introduced in the middle of 1988 to a wider user community, and work
on applications could start. A service for electronic mail was
developed, a graphics system was added, and various efforts for
general document preparation systems proceeded. The display facility
was extended to accommodate two screens, including color. At the same
time, feedback from experience in its use was incorporated by
improving existing parts. Since 1989, Oberon has replaced Modula-2 in
our introductory programming courses.

\beginchapter CHAPTER 2: STRUCTURE OF THE SYSTEM.

\medskip

\beginsubsection 2.1. INTRODUCTION.

In order to warrant the sizeable effort of designing and constructing
an entire operating system from scratch, a number of basic concepts
need to be novel. We start this chapter with a discussion of the
principal concepts underlying the Oberon System and of the dominant
design decisions. On this basis, a presentation of the system's
structure follows. It will be restricted to its coarsest level, namely
the composition and interdependence of the largest building blocks,
the modules. The chapter ends with an overview of the remainder of the
book. It should help the reader to understand the role, place, and
significance of the parts described in the individual chapters.

The fundamental objective of an operating system is to present the
computer to the user and to the programmer at a certain level of
abstraction. For example, the store is presented in terms of
requestable pieces or variables of a specified data type, the disk is
presented in terms of sequences of characters (or bytes) called files,
the display is presented as rectangular areas called viewers, the
keyboard is presented as an input stream of characters, and the mouse
appears as a pair of coordinates and a set of key states. Every
abstraction is characterized by certain properties and governed by a
set of operations. It is the task of the system to implement these
operations and to manage them, constrained by the available resources
of the underlying computer. This is commonly called resource
management.

Every abstraction inherently hides details, namely those from which it
abstracts. Hiding may occur at different levels. For example, the
computer may allow certain parts of the store, or certain devices to
be made inaccessible according to its mode of operation
(user/supervisor mode), or the programming language may make certain
parts inaccessible through a hiding facility inherent in its
visibility rules. The latter is of course much more flexible and
powerful, and the former indeed plays an almost negligible role in our
system. Hiding is important because it allows maintenance of certain
properties (called {\it invariants\/}) of an abstraction to be
guaranteed. Abstraction is indeed the key of any modularization, and
without modularization every hope of being able to guarantee
reliability and correctness vanishes. Clearly, the Oberon System was
designed with the goal of establishing a modular structure on the
basis of purpose-oriented abstractions. The availability of an
appropriate programming language is an indispensable prerequisite, and
the importance of its choice cannot be over-emphasized.

\beginsubsection 2.2. CONCEPTS.

\beginsubsubsection 2.2.1. Viewers.

Whereas the abstractions of individual variables representing parts of
the primary store, and of files representing parts of the disk store
are well established notions and have significance in every computer
system, abstractions regarding input and output devices became
important with the advent of high interactivity between user and
computer. High interactivity requires high bandwidth, and the only
channel of human users with high bandwidth is the eye. Consequently,
the computer's visual output unit must be properly matched with the
human eye. This occurred with the advent of the high-resolution
display in the mid 1970s, which in turn had become feasible due to
faster and cheaper electronic memory components. The high-resolution
display marked one of the few very significant break-throughs in the
history of computer development. The typical bandwidth of a modern
display is in the order of 100 MHz. Primarily the high-resolution
display made visual output a subject of abstraction and {\it resource management\/}.
In the Oberon System, the display is partitioned into
{\it viewers\/}, also called {\it windows\/}, or more precisely, into {\it frames\/},
rectangular areas of the screen(s). A viewer typically consists of two
frames, a title bar containing a subject name and a menu of commands,
and a main frame containing some text, graphic, picture, or other
object. A viewer itself is a frame; frames can be nested, in principle
to any depth.

The System provides routines for generating a frame (viewer), for
moving and for closing it. It allocates a new viewer at a specified
place, and upon request delivers hints as to where it might best be
placed. It keeps track of the set of opened viewers. This is what is
called {\it viewer management\/}, in contrast to the handling of their
displayed contents.

But high interactivity requires not only a high bandwidth for visual
output, it demands also flexibility of input. Surely, there is no need
for an equally large bandwidth, but a keyboard limited by the speed of
typing to about 100 Hz is not good enough. The break-through on this
front was achieved by the so-called {\it mouse\/}, a pointing device which
appeared roughly at the same time as the high-resolution display.

This was by no means just a lucky coincidence. The mouse comes to
fruition only through appropriate software and the high-resolution
display. It is itself a conceptually very simple device delivering
signals when moved on the table. These signals allow the computer to
update the position of a mark--- the cursor ---on the display. Since
feedback occurs through the human eye, no great precision is required
from the mouse. For example, when the user wishes to identify a
certain object on the screen, such as a letter, he moves the mouse as
long as required until the mapped cursor reaches the object. This
stands in marked contrast to a digitizer which is supposed to deliver
exact coordinates. The Oberon System relies very much on the
availability of a mouse.

Perhaps the cleverest idea was to equip mice with buttons. By being
able to signal a request with the same hand that determines the cursor
position, the user obtains the direct impression of issuing
position-dependent requests. Position-dependence is realized in
software by delegating interpretation of the signal to a procedure---
a so-called {\it handler\/} or interpreter ---which is local to the viewer in
whose area the cursor momentarily appears. A surprising flexibility of
command activation can be achieved in this manner by appropriate
software. Various techniques have emerged in this connection,
e.g. pop-up menus, pull-down-menus, etc. which are powerful even under
the presence of a single button only. For many applications, a mouse
with several keys is far superior, and the Oberon System basically
assumes three buttons to be available. The assignment of different
functions to the keys may of course easily lead to confusion when
every application prescribes different key assignment. This is,
however, easily avoided by the adherence to certain ``global''
conventions. In the Oberon System, the left button is primarily used
for {\it marking\/} a position (setting a caret), the middle button for
issuing general {\it commands\/} (see below), and the right button for
{\it selecting\/} displayed objects.

Recently, it has become fashionable to use overlapping windows
mirroring documents being piled up on one's desk. We have found this
metaphor not entirely convincing. Partially hidden windows are
typically brought to the top and made fully visible before any
operation is applied to their contents. In contrast to the
insignificant advantage stands the substantial effort necessary to
implement this scheme. It is a good example of a case where the
benefit of a complication is incommensurate with its cost. Therefore,
we have chosen a solution that is much simpler to realize, yet has no
genuine disadvantages compared to overlapping windows: tiled viewers
as shown in Fig. 2.1.


\beginsubsubsection 2.2.2. Commands.

Position-dependent commands with fixed meaning (fixed
for each type of viewer) must be supplemented by general
commands. Conventionally, such commands are issued through the
keyboard by typing the program's name that is to be executed into a
special command text. In this respect, the Oberon System offers a
novel and much more flexible solution which is presented in the
following paragraphs.  First of all we remark that a program in the
common sense of a text compiled as a unit is mostly a far too large
unit of action to serve as a command. Compare it, for example, with
the insertion of a piece of text through a mouse command. In Oberon,
the notion of a unit of action is separated from the notion of unit of
compilation. The former is a command represented by a (exported)
procedure, the latter is a module. Hence, a module may, and typically
does, define several, even many commands. Such a (general) command may
be invoked at any time by pointing at its name in any

text visible in any viewer on the display, and by clicking the middle
mouse button. The command name has the form {\tt M.P}, where {\tt P} is the
procedure's identifier and {\tt M} that of the module in which {\tt P} is
declared. As a consequence, any command click may cause the loading of
one or several modules, if {\tt M} is not already present in main store. The
next invocation of {\tt M.P} occurs instantaneously, since {\tt M} is already
loaded. A further consequence is that modules are never
(automatically) removed, because a next command may well refer to the
same module.

TODO: Fig. 2.1. Oberon display with tiled viewers

Every command has the purpose to alter the state of some
operands. Typically, they are denoted by text following the command
identification, and Oberon follows this convention. Strictly speaking,
commands are denoted as parameterless procedures; but the system
provides a way for the procedure to identify the text position of its
origin, and hence to read and interpret the text following the
command, i.e. the actual parameters. Both reading and interpretation
must, however, be programmed explicitly.

The parameter text must refer to objects that exist before command
execution starts and are quite likely the result of a previous command
interpretation. In most operating systems, these objects are {\it files\/}
registered in the directory, and they act as interfaces between
commands. The Oberon System broadens this notion; the links between
consecutive commands are not restricted to files, but can be any
global variable, because modules do not disappear from storage after
command termination, as mentioned above.

This tremendous flexibility seems to open Pandora's box, and indeed it
does when misused. The reason is that global variables' states may
completely determine and alter the effect of a command. The variables
represent {\it hidden states\/}, hidden in the sense that the user is in
general unaware of them and has no easy way to determine their
value. The positive aspect of using global variables as interfaces
between commands is that some of them may well be visible on the
display. All viewers---and with them also their contents---are
organized in a data structure that is rooted in a global variable (in
module {\it Viewers\/}). Parts of this variable therefore constitute {\it visible states\/},
and it is highly appropriate to refer to them as command
parameters.

One of the rules of what may be called the Oberon Programming Style is
therefore to avoid hidden states, and to reduce the introduction of
global variables. We do not, however, raise this rule to the rank of a
dogma. There exist genuinely useful exceptions, even if the variables
have no visible parts.

There remains the question of how to denote visible objects as command
parameters. An obvious case is the use of the most recent selection as
parameter. A procedure for locating that selection is provided by
module Oberon. (It is restricted to text selections). Another
possibility is the use of the caret position in a text. This is used
in the case of inserting new text; the pressing of a key on the
keyboard is also considered to be a command, and it causes the
character's insertion at the caret position.

A special facility is introduced for designating viewers as operands:
the star marker. It is placed at the cursor position when the
keyboard's mark key ({\tt SETUP}) is pressed. The procedure
{\tt Oberon.MarkedViewer} identifies the viewer in whose area the star
lies. Commands which take it as their parameter are typically followed
by an asterisk in the text. Whether the text contained in a text
viewer, or a graph contained in a graphic viewer, or any other part of
the marked viewer is taken as the actual parameter depends on how the
command procedure is programmed.

Finally, a most welcome property of the system should not remain
unmentioned. It is a direct consequence of the persistent nature of
global variables and becomes manifest when a command fails. Detected
failures result in a trap. Such a trap should be regarded as an
abnormal command termination. In the worst case, global data may be
left in an inconsistent state, but they are not lost, and a next
command can be initiated based on their current state. A trap opens a
small viewer and lists the sequence of invoked procedures with their
local variables and current values. This information helps a
programmer to identify the cause of the trap.

\beginsubsubsection 2.2.3. Tasks.

From the presentations above it follows that the Oberon System is
distinguished by a highly flexible scheme of command activation. The
notion of a command extends from the insertion of a single character
and the setting of a marker to computations that may take hours or
days. It is moreover distinguished by a highly flexible notion of
operand selection not restricted to registered, named files. And most
importantly, by the virtual absence of hidden states. The state of the
system is practically determined by what is visible to the user.

This makes it unnecessary to remember a long history of previously
activated commands, started programs, entered modes, etc. Modes are in
our view the hallmark of user-unfriendly systems. It should at this
point have become obvious that the system allows a user to pursue
several different tasks concurrently. They are manifest in the form of
viewers containing texts, graphics, or other displayable objects. The
user switches between tasks implicitly when choosing a different
viewer as operand for the next command. The characteristic of this
concept is that task switching is under explicit control of the user,
and the atomic units of action are the commands.

At the same time, we classify Oberon as a single-process (or
single-thread) system. How is this apparent paradox to be understood?
Perhaps it is best explained by considering the basic mode of
operation. Unless engaged in the interpretation of a command, the
processor is engaged in a loop continuously polling event
sources. This loop is called the {\it central loop\/}; it is contained in
module {\tt Oberon} which may be regarded as the system's heart. The two
fixed event sources are the mouse and the keyboard. If a keyboard
event is sensed, control is dispatched to the handler installed in the
so-called {\it focus viewer\/}, designated as the one holding the caret. If a
mouse event (key) is sensed, control is dispatched to the handler in
which the cursor currently lies. This is all possible under the
paradigm of a single, uninterruptible process.

The notion of a single process implies non-interruptability, and
therefore also that commands cannot interact with the
user. Interaction is confined to the selection of commands before
their execution. Hence, there exists no input statement in typical
Oberon programs. Inputs are given by parameters supplied and
designated before command invocation.

This scheme at first appears as gravely restrictive. In practice it is
not, if one considers single-user operation. It is this single user
who carries out a dialog with the computer. A human might be capable
of engaging in simultaneous dialogs with several processes only if the
commands issued are very time-consuming. We suggest that execution of
time-consuming computations might better be delegated to loosely
coupled compute-servers in a distributed system.

The primary advantage of a system dealing with a single process is
that task switches occur at user-defined points only, where no local
process state has to be preserved until resumption. Furthermore,
because the switches are user-chosen, the tasks cannot interfere in
unexpected and uncontrollable ways by accessing common variables. The
system designer can therefore omit all kinds of protection mechanisms
that exclude such interference. This is a significant simplification.

The essential difference between Oberon and multiprocess-systems is
that in the former task switches occur between commands only, whereas
in the latter a switch may be invoked after any single
instruction. Evidently, the difference is one of granularity of
action. Oberon's granularity is coarse, which is entirely acceptable
for a single-user system.

The system offers the possibility to insert further polling commands
in the central loop. This is necessary if additional event sources are
to be introduced. The prominent example is a network, where commands
may be sent from other workstations. The central loop scans a list of
so-called {\it task descriptors\/}. Each descriptor refers to a command
procedure. The two standard events are selected only if their guard
permits, i.e. if either keyboard input is present, or if a mouse event
occurs. Inserted tasks must provide their own guard in the beginning
of the installed procedure.

The example of a network inserting commands, called {\it requests\/}, raises a
question: what happens if the processor is engaged in the execution of
another command when the request arrives? Evidently, the request would
be lost unless measures are taken. The problem is easily remedied by
buffering the input. This is done in every driver of an input device,
in the keyboard driver as well as the network driver. The incoming
signal triggers an interrupt, and the invoked interrupt handler
accepts the input and buffers it. We emphasize that such interrupt
handling is confined to drivers, system components at the lowest
level. An interrupt does not evoke a task selection and a task
switch. Control simply returns to the point of interruption, and the
interrupt remains unnoticeable to programs. There exists, as with
every rule, an exception: an interrupt due to keyboard input of the
abort character returns control to the central loop.

\beginsubsubsection 2.2.4. Tool Texts as Configurable Menus.

Certainly, the concepts of viewers specifying their own interpretation
of mouse clicks, of commands invokable from any text on the display,
of any displayed object being selectable as an interface between
commands, and of commands being dialog-free, uninterruptible units of
action, have considerable influence on the style of programming in
Oberon, and they thoroughly change the style of using the
computer. The ease and flexibility in the way pieces of text can be
selected, moved, copied, and designated as command and as command
parameters, drastically reduces the need for typing. The mouse becomes
the dominant input device: the keyboard merely serves to input textual
data. This is accentuated by the use of so-called {\it tool texts\/},
compositions of frequently used commands, which are typically
displayed in the narrower system track of viewers. One simply doesn't
type commands! They are usually visible somewhere already. Typically,
the user composes a tool text for every project pursued. Tool texts
can be regarded as individually configurable private menus.

The rarity of issuing commands by typing them has the most agreeable
benefit that their names can be meaningful words. For example, the
copy operation is denoted by {\tt Copy} instead of {\tt cp}, rename by {\tt Rename}
instead of {\tt rn}, the call for a file directory excerpt is named
{\tt Directory} instead of {\tt ls}. The need for memorizing an infinite list of
cryptic abbreviations, which is another hallmark of user-unfriendly
systems, vanishes.

But the influence of the Oberon concept is not restricted to the style
in which the computer is used. It extends also to the way programs are
designed to communicate with the environment. The definition of the
abstract type {\tt Text} in the system's core suggests the replacement
of files by texts as carrier of input and output data in very many
cases. The advantage to be gained lies in the text's immediate
editability. For example, the output of the command {\tt System.Directory}
produces the desired excerpt of the file directory
in the form of a (displayed) text. Parts of it or the whole may be
selected and copied into other texts by regular editing commands
(mouse clicks). Or, the compiler accepts texts as input. It is
therefore possible to compile a text, execute the program, and to
recompile the re-edited text without storing it on disk between
compilations and tests. The ubiquitous editability of text together
with the persistence of global data (in particular viewers) allows
many steps that do not contribute to the progress of the task actually
pursued to be avoided.

\beginsubsubsection 2.2.5. Extensibility.

An important objective in the design of the Oberon System was
extensibility. It should be easy to extend the system with new
facilities by adding modules that make use of the already existing
resources. Equally important, it should also reduce the system to
those facilities that are currently and actually used. For example, a
document editor processing documents free of graphics should not
require the loading of an extensive graphics editor, a workstation
operating as a stand-alone system should not require the loading of
extensive network software, and a system used for clerical purposes
need include neither compiler nor assembler. Also, a system
introducing a new kind of display frame should not include procedures
for managing viewers containing such frames. Instead, it should make
use of existing viewer management. The staggering consumption of
memory space by many widely used systems is due to violation of such
fundamental rules of engineering. The requirement of many megabytes of
store for an operating system is, albeit commonly tolerated, absurd
and another hallmark of user-unfriendliness, or perhaps manufacturer
friendliness. Its reason is none other than inadequate extensibility.

We do not restrict this notion to procedural extensibility, which is
easy to realize. The important point is that extensions may not only
add further procedures and functions, but introduce their own data
types built on the basis of those provided by the system: data
extensibility. For example, a graphics system should be able to define
its graphics frames based on frames provided by the basic display
module and by extending them with attributes appropriate for graphics.


This requires an adequate language feature. The language Oberon
provides precisely this facility in the form of {\it type extensions\/}. The
language was designed for this reason; Modula-2 would have been the
choice, had it not been for the lack of a type extension feature. Its
influence on system structure was profound, and the results have been
most encouraging. In the meantime, many additions have been created
with surprising ease. One of them is described at the end of this
book. The basic system is nevertheless quite modest in its resource
requirements (see Table at the end of Section 2.3).

\beginsubsubsection 2.2.6. Dynamic Loading.

Activation of commands residing in modules that are not present in the
store implies the loading of the modules and, of course, all their
imports. Invoking the loader is, however, not restricted to command
activation; it may also occur through programmed procedure calls. This
facility is indispensable for a successful realization of genuine
extensibility. Modules must be loadable on demand. For example, a
document editor loads a graphics package when a graphic element
appears in the processed document, but not otherwise.

The Oberon System features no separate linker. A module is linked with
its imports when it is loaded, and never before. As a consequence,
every module is present only once, in main store (linked) as well as
on backing store (unlinked, as file). Avoiding the generation of
multiple copies in different, linked object files is the key to
storage economy. Prelinked mega-files do not occur in the Oberon
System, and every module is freely reusable.

\beginsubsection 2.3. THE SYSTEM'S STRUCTURE.


The largest identifiable units of the system are its modules. It is
therefore most appropriate to describe a system's structure in terms
of its modules. As their interfaces are explicitly declared, it is
also easy to exhibit their interdependence in the form of a directed
graph. The edges indicate imports. The module graph of a system
programmed in Oberon is hierarchical, i.e. has no cycles. The lowest
members of the hierarchy effectively import hardware only. We refer
here to modules which contain device drivers. But module Kernel also
belongs to this class; it ``imports memory'' and includes the disk
driver. The modules on the top of the hierarchy effectively export to
the user. As the user has direct access to command procedures, we call
these top members {\it command modules\/} or tool modules.

The hierarchy of the basic system is shown in a table of direct
imports and as a graph in Figure 2.2. The picture is simplified by
omitting direct import edges if an indirect path also leads from the
source to the destination. For example, {\tt Files\/} imports {\tt Kernel\/}; the
direct import is not shown, because a path from {\tt Kernel\/} leads to {\tt Files\/}
via {\tt FileDir\/}.

TODO: Insert Figure 2.2

Module names in the plural form typically indicate the definition of
an abstract data type in the module. The type is exported together
with the pertinent operations. Examples are {\tt Files\/}, {\tt
  Modules\/}, {\tt Fonts\/}, {\tt Texts\/}, {\tt Viewers\/}, {\tt
  MenuViewers\/}, and {\tt TextFrames\/}. Modules whose names are in
singular form typically denote a resource that the module manages, be
it a global variable or a device. The variable or the device is itself
hidden (not exported) and becomes accessible through the module's
exported procedures. Examples are all device drivers, {\tt Input\/}
for keyboard and mouse, {\tt Kernel\/} for memory and disk, and {\tt
  Display\/}. Exceptions are the command modules whose name is mostly
chosen according to the activity they primarily represent, like {\tt
  System\/}, and {\tt Edit\/}.

Module {\tt Oberon\/} is, as already mentioned, the heart of the system
containing the central loop to which control returns after each
command interpretation, independent of whether it terminates normally
or abnormally. {\tt Oberon\/} exports several procedures of auxiliary nature,
but primarily also the one allowing the invocation of commands ({\tt Call\/})
and access to the command's parameter text through variable
{\tt Oberon.Par}. Furthermore, it contains global, exported variables: the
log text. The log text typically serves to issue prompts and short
failure reports of commands. The text is displayed in a log viewer
that is automatically opened when module {\tt System\/} is initialized. Module
{\tt Oberon\/} furthermore contains the two markers used globally on the
display, the {\it mouse cursor\/} and the {\it star pointer\/}. It exports procedures
to draw and to erase them, and allows the installation of different
patterns for them.

The system shown in Fig. 2.2. basically contains facilities for
generating and editing texts, and for storing them in the file
system. All other functions are performed by modules that must be
added in the usual way by module loading on demand. This includes,
notably, the compiler, network communication, document editors, and
all sorts of programs designed by users. The high priority given in
the system's conception to modularity, to avoiding unnecessary frills,
and to concentrate on the indispensable in the core, has resulted in a
system of remarkable compactness. Although this property may be
regarded as of little importance in this era of falling costs of large
memories, we consider it to be highly essential. We merely should like
to draw the reader's attention to the correlation between a systems'
size and its reliability. Also, we do not consider it as good
engineering practice to consume a resource lavishly just because it
happens to be cheap. The following table lists the core's modules and
the major application modules, and it indicates the size of code (in
words) and static variables (in bytes) and, the number of source
program lines.

TODO: Table here

\beginsubsection 2.4. A TOUR THROUGH THE CHAPTERS.

Implementation of a system proceeds bottom-up. Naturally, because
modules on higher levels are clients of those on the lower levels and
cannot function without the availability of their imports. Description
of a system, on the other hand, is better ordered in the top-down
direction. This is because a system is designed with its expected
applications and functions in mind. Decomposition into a hierarchy of
modules is justified by the use of auxiliary functions and
abstractions and by postponing their more detailed explanation to a
later time when their need has been fully motivated. For this reason,
we will proceed essentially in the top-down direction.

Chapters 3--5 describe the outer core of the system. Chapter 3
focusses on the dynamic aspects. In particular, this chapter
introduces the fundamental operational units of {\it task\/} and
{\it command\/}. Oberon's tasking model distinguishes the categories of
{\it interactiv taskse\/} and {\it background tasks\/}. Interactive tasks are
represented on the display screen by rectangular areas, so-called
{\it viewers\/}. Background tasks need not be connected with any displayed
object. They are scheduled with low priority when interactions are
absent. A good example of a background task is the memory garbage
collector. Both interactive tasks and background tasks are mapped to a
single process by the task scheduler. Commands in Oberon are explicit,
atomic units of interactive operations. They are realized in the form
of exported parameterless procedures and replace the heavier-weight
notion of program known from more conventional operating systems. This
chapter continues with a definition of a software toolbox as a
logically connected collection of commands. It terminates with an
outline of the system control toolbox.


Chapter 4 explains Oberon's display system. It starts with a
discussion of our choice of a hierarchical tiling strategy for the
allocation of viewers. A detailed study of the exact role of Oberon
viewers follows. Type {\tt Viewer\/} is presented as an object class with an
open message interface providing a conceptual basis for far-reaching
extensibility. Viewers are then recognized as just a special case of
so-called {\it frames\/} that may be nested. A category of standard viewers
containing a menu frame and a frame of contents is investigated. The
next topic is cursor handling. A cursor in Oberon is a marked
path. Both viewer manager and cursor handler operate on an abstract
logical display area rather than on individual physical monitors. This
allows a unified handling of display requests, independent of number
and types of monitors assigned. For example, smooth transitions of the
cursor across screen boundaries are conceptually guaranteed. The
chapter continues with the presentation of a concise and complete set
of raster operations that is used to place textual and graphical
elements in the display area. An overview of the system display
toolbox concludes the chapter.

Chapter 5 introduces text. Oberon distinguishes itself by treating
Text as an abstract data type that is integrated in the central
system. Numerous fundamental consequences are discussed. For example,
a text can be produced by one command, edited by a user, and then
consumed by a next command. Commands themselves can be represented
textually in the form {\tt M.P\/}o, followed by a textual parameter
list. Consequently, any command can be called directly from within a
text (so-called {\it tool\/}) simply by pointing at it with the
mouse. However, the core of this chapter is a presentation of Oberon's
text system as a case study in program modularization. The concerns of
managing a text and displaying it are nicely separated. Both the text
manager and the text display feature an abstract public interface as
well as an internally hidden data structure. Finally in this chapter,
Oberon's type-font management and the toolbox for editing are
discussed.

Chapters 6--9 describe the {\it inner core\/}, still in a top-down
path. Chapter~6 explains the loader of program modules and motivates
the introduction of the data type {\tt Module\/}. The chapter includes the
management of the memory part holding program code and defines the
format in which compiled modules are stored as object
files. Furthermore, it discusses the problems of binding separately
compiled modules together and of referencing objects defined in other
modules.

Chapter 7 is devoted to the file system, a part of crucial importance,
because files are involved in almost every program and
computation. The chapter consist of two distinct parts, the first
introducing the type {\tt File\/} and describing the structure of files,
i.e. their representation on disk storage with its sequential
characteristics, the second describing the directory of file names and
its organisation as a B-tree for obtaining fast searches.

The management of memory is the subject of Chapter 8. A single,
central storage management was one of the key design decisions,
guaranteeing an efficient and economical use of storage. The chapter
explains the store's partitioning into specific areas. Its central
concern, however, is the discussion of dynamic storage management in
the partition called the {\it heap\/}. The algorithm for allocation
(corresponding to the intrinsic procedure {\tt NEW\/}) and for retrieval
(called garbage collection) are explained in detail.

At the lowest level of the module hierarchy we find device
drivers. They are described in Chapter 9, which contains drivers for
some widely accepted interface standards. The first is PS-2, a serial
transmission with synchronous clock. This is used for the keyboard and
for the Mouse. The second is SPI, a standard for bi-directional,
serial transmission with synchronous clock. This is used for the
``disk'', represented by an SDI-card (flash memory), and for the
network. And the third standard is RS-232 typically used for simple
and slow data links. It is bidirectional and asynchronous.

The second part of the book, consisting of Chapters 10--15, is
devoted to what may be called first applications of the basic Oberon
System. These chapters are therefore independent of each other, making
reference to Chapters 3--9 only.

Although the Oberon System is well-suited for operating stand-alone
workstations, a facility for connecting a set of computers should be
considered as fundamental. Module {\tt Net\/}, which makes transmission of
files among workstations connected by a bus-like network possible, is
the subject of Chapter 10. It presents not only the problems of
network access, of transmission failures and collisions, but also
those of naming partners. The solutions are implemented in a
surprisingly compact module which uses a network driver presented in
Chapter 9.

When a set of workstations is connected in a network, the desire for a
central server appears. A central facility serving as a file
distribution service, as a printing station, and as a storage for
electronic mail is presented in Chapter 11. It emerges by extending
the {\tt Net\/} module of Chapter 10, and is a convincing application of the
tasking facilities explained in Section 2.2. In passing we note that
the server operates on a machine that is not under observation by a
user. This circumstance requires an increased degree of robustness,
not only against transmission failures, but also against data that do
not conform to defined formats.

The presented system of servers demonstrates that Oberon's
single-thread scheme need not be restricted to single-user
systems. The fact that every command or request, once accepted, is
processed until completion, is acceptable if the request does not
occupy the processor for too long, which is mostly the case in the
presented server applications. Requests arriving when the processor is
engaged are queued. Hence, the processor handles requests one at a
time instead of interleaving them which, in general, results in faster
overall performance due to the absence of frequent task switching.

Chapter 12 describes the Oberon compiler. It translates source text in
Oberon into target code, i.e. instruction sequences of some target
computer. Its principles and techniques are explained in [6]. Both,
source language and target architecture must be understood before
studying a compiler. Both source language and the target computer's
RISC architecture are presented in the Appendix.

Although here the compiler appears as an application module, it
naturally plays a distinguished role, because the system (and the
compiler itself) is formulated in the language which the compiler
translates into code. Together with the text editor it was the
principal tool in the system's development. The use of
straight-forward algorithms for parsing and symbol table organization
led to a reasonably compact piece of software. A main contributor to
this result is the language's definition: the language is devoid of
complicated structures and rarely used embellishments.

The compiler and thereby the chapter is partitioned into two main
parts. The first is language- specific, but does not refer to any
particular target computer. It consist of the {\it scanner\/} and the
{\it parser\/}. This part is therefore of most general interest to the
readership. The second part is, essentially, language-independent, but
is specifically tailored to the instruction set of the target
computer. It is called the {\it code generator\/}.

Texts play a predominant role in the Oberon System. Their preparation
is supported by the system's major tool, the editor. In Chapter 13 we
describe another editor, one that handles graphic objects. At first,
only horizontal and vertical lines and short captions are introduced
as objects. The major difference to texts lies in the fact that their
coordinates in the drawing plane do not follow from those of their
predecessor automatically, because they form a set rather than a
sequence. Each object carries its own, independent coordinates. The
influence of this seemingly small difference upon an editor are
far-reaching and permeate the entire design. There exist hardly any
similarities between a text and a graphics editor. Perhaps one should
be mentioned: the partitioning into three parts. The bottom module
defines the respective abstract data structure for texts or graphics,
together with, of course, the procedures handling the structure, such
as searches, insertions, and deletions. The middle module in the
hierarchy defines a respective frame and contains all procedures
concerned with displaying the respective objects including the frame
handler defining interpretation of mouse and keyboard events. The top
modules are the respective tool modules ({\tt Edit\/}, {\tt Draw\/}). The presented
graphics editor is particularly interesting in so far as it
constitutes a convincing example of Oberon's extensibility. The
graphics editor is integrated into the entire system; it embeds its
graphic frames into menu-viewers and uses the facilities of the text
system for its caption elements. And lastly, new kinds of elements can
be incorporated by the mere addition of new modules, i.e. without
expanding, even without recompiling the existing ones. Two examples
are shown in Chapter 13 itself: rectangles and circles.

The Draw System has been extensively used for the preparation of
diagrams of electronic circuits. This application suggests a concept
that is useful elsewhere too, namely a recursive definition of the
notion of object. A set of objects may be regarded as an object itself
and be given a name. Such an object is called a {\it macro\/}. It is a
challenge to the designer to implement a macro facility such that it
is also extensible, i.e. in no way refers to the type of its elements,
not even in its input operations of files on which macros are stored.

Chapter 14 presents two other tools, namely one used for installing an
Oberon System on a bare machine, and one used to recover from failures
of the file store. Although rarely employed, the first was
indispensable for the development of the system. The maintenance or
recovery tools are invaluable assets when failures occur. And they do!
Chapter 14 covers material that is rarely presented in the literature.

Chapter 15 is devoted to tools that are not used by the Oberon System
presented so far, but may be essential in some applications. The first
is a data link with a protocol based on the RS-232 standard shown in
Chapter 9. Another is a standard set of basic mathematical
functions. And the third is a tool for creating new macros for the
Draw System.

The third part of this book is devoted to a detailed description of
the hardware. Chapter 16 defines the processor, for which the compiler
generates code. The target computer is a truly simple and regular
processor called RISC with only 14 instructions, represented not by a
commercial processor, but implemented with an FPGA, a Field
Programmable Gate Array. It allows its structure to be described in
full detail. It is a straight-forward, von Neumann type device
consisting of a register bank, an arithmetic-logic unit, including a
floating-point unit. Typical optimization facilities, like pipelining
and cache memory, have been omitted for the sake of transparency and
simplicity. The processor circuit is described in the language
{\it Verilog\/}.

Chapter 17 describes the environment in which the processor is
embedded. This environment consists of the interfaces to main memory
and to all external devices.

TODO: References
1. N. Wirth. The programming language Oberon. Software - Practice and Experience 18, 7, (July 1988) 671-690.
2. M. Reiser and N. Wirth. Programming in Oberon - Steps beyond Pascal and Modula. Addison- Wesley, 1992. ISBN 0-201-56543-9
3. N. Wirth and J. Gutknecht. The Oberon System. Software - Practice and Experience, 19, 9 (Sept. 1989), 857-893.
4. N. Wirth. Ceres-Net: A low-cost computer network. Software - Practice and Experience, 20, 1 (Jan. 1990), 13-24.
5. M. Reiser. The Oberon System - User Guide and Programmer's Manual. Addison-Wesley, 1991. ISBN 0-201-54422-9
6. N. Wirth. Compiler Construction. Addison-Wesley, Reading, 1996. ISBN 0-201-40353-6

\beginchapter CHAPTER 3: THE TASKING SYSTEM.

Eventually, it is the generic ability to perform every conceivable
task that turns a computing device into a versatile universal
tool. Consequently, the issues of modeling and orchestrating of tasks
are fundamental in the design of any operating system. Of course, we
cannot expect a single fixed tasking metaphor to be the ideal solution
for all possible kinds of systems and modes of use. For example,
different metaphors are probably appropriate in the cases of a closed
mainframe system serving a large set of users in time-sharing mode on
the one hand, and of a personal workstation that is operated by a
single user at a high degree of interactivity on the other hand.

In the case of Oberon, we have consciously concentrated on the domain
of personal workstations. More precisely, we have directed Oberon's
tasking facilities towards a single-user interactive personal
workstation that is possibly integrated into a local area network.

We start the presentation in Section 3.1 with a clarification of the
technical notion of task. In Section 3.2, we continue with a detailed
explanation of the scheduling strategy. Then, in Section 3.3, we
introduce the concept of {\it command\/}. And finally, Section 3.4 provides an
overview of predefined system-oriented toolboxes, i. e. coherent
collections of commands devoted to some specific topic. Example topics
are system control and diagnosis, display management, and file
management.

\beginsubsection 3.1. THE CONCEPT OF TASK.

In principle, we distinguish two categories of tasks in Oberon: {\it
  Interactive tasks\/} and {\it background tasks\/}. Loosely speaking,
interactive tasks are bound to local regions on the display screen and
to interactions with their contents while, in contrast, background
tasks are system-wide and not necessarily related to any specific
displayed entity.

\beginsubsubsection 3.1.1. Interactive tasks.

Every interactive task is represented by a so-called {\it viewer\/}. Viewers
constitute the interface to Oberon's display-system. They embody a
variety of roles that are collected in an abstract data type
{\tt Viewer\/}. We shall give a deeper insight into the display system in
Chapter 4. For the moment it suffices to know that viewers are
represented graphically as rectangles on the display screen and that
they are implicit carriers of interactive tasks. Figure 3.1 shows a
typical Oberon display screen that is divided up into seven viewers
corresponding to seven simultaneously active interactive tasks.

In order to get firmer ground under our feet, we now present the programmed declaration of type {\tt Viewer\/} in a slightly abstracted form:
\begintt
Viewer = POINTER TO ViewerDesc;

ViewerDesc = RECORD X, Y, W, H: INTEGER;
  handle: Handler;
  state: INTEGER;
END;
\endtt

\noindent {\tt X\/}, {\tt Y\/}, {\tt W\/}, {\tt H\/} define the viewer's rectangle on the screen, i.e. location
{\tt X\/}, {\tt Y\/} of the lower left corner relative to the display origin, width {\tt W\/}
and height {\tt H\/}. The variable {\tt state\/} informs about the current state of
visibility (visible, closed, covered), while {\tt handle\/} represents the
functional interface of viewers. The type of the handler is
\begintt
Handler = PROCEDURE (V: Viewer; VAR M: ViewerMsg);
\endtt

\noindent where ViewerMsg is some base type of messages whose exact declaration is of minor importance for the moment:
\begintt
ViewerMsg = RECORD ... (*basic parameter fields*) END;
\endtt

TODO: Figure 3.1 Typical Oberon display configuration with tool track on the right

However, we should point out the use of object-oriented
terminology. It is justified because {\tt handle\/} is a procedure variable (a
{\it handler\/}) whose identity depends on the specific viewer. A call
{\tt V.handle(V, M)} can therefore be interpreted as the sending of a
message {\tt M\/} to be handled by the method of the receiving viewer {\tt V\/}.

We recognize an important difference between the standard
object-oriented model and our handler paradigm. The standard model is
closed in the sense that only a fixed set of messages is understood by
a given class of objects. In contrast, the handler paradigm is {\it open\/}
because it defines just the root ({\tt ViewerMsg\/}) of a potentially
unlimited tree of extending message types. For example, a concrete
handler might be able to handle messages of type {\tt MyViewerMsg\/}, where
\begintt
MyViewerMsg = RECORD (ViewerMsg)
  mypar: MyParameters
END;
\endtt

\noindent is an extended type of {\tt ViewerMsg\/}.

It is worth noting that our open object-oriented model is extremely
flexible. Notably, extending the set of message types that are handled
by an object is a mere implementation issue, that is, it has no effect
at all on the object’s compile-time interface and on the system
integrity. It is fair to mention though that such a high degree of
extensibility does not come for free. The price to pay is the
obligation of explicit message {\it dispatching at runtime\/}. The following
Chapters will capitalize on this property.

Coming back to the perspective of tasks, we note that each sending of
a message to a viewer corresponds to an activation or reactivation of
the interactive task that it represents.

\beginsubsubsection 3.1.2. Background Tasks.

Oberon background tasks are not connected a priori with any specific
aggregate in the system. Seen technically, they are instances of an
abstract data type consisting of type declarations Task and TaskDesc
together with intrinsic operations {\tt NewTask\/}, {\tt Install\/} and {\tt Remove\/}:
\begintt
Task = POINTER TO TaskDesc;
TaskDesc = RECORD state: INTEGER; handle: PROCEDURE END;

PROCEDURE NewTask(h: PROCEDURE; period: INTEGER): Task;
PROCEDURE Install (T: Task);
PROCEDURE Remove (T: Task);
\endtt

\noindent The procedures {\tt Install\/} and {\tt Remove\/} are called explicitly in order to
transfer the state of the specified task from {\tt offline\/} to {\tt idle\/} and from
{\tt idle\/} to {\tt offline\/} respectively. Installed tasks take their turns in
becoming {\tt active\/}, that is, in being executed. The installed handlers
are simple, parameterless procedures specifying their own actions and
conditions for execution, with one exception: Resumption may be
delayed until a certain period of time has elapsed. This period is
specified in milliseconds when a task is created.

The following two examples of concrete background tasks may serve a
better understanding of our explanations. The first one is a
system-wide garbage collector collecting unused memory. The second
example is a network monitor accepting incoming data on a local area
network. In both examples the state of the task is captured entirely
by global system variables. We shall come back to these topics in
Chapters 8 and 10 respectively.

We should not end this Section without drawing an important
conclusion. Transfers of control between tasks are implemented in
Oberon as ordinary calls and returns of ordinary procedures (procedure
variables, actually). Preemption is not possible. From that we
conclude that active periods of tasks are sequentially ordered and can
be controlled by a single thread of control. This simplification pays
well: Locks of common resources are completely dispensable and
deadlocks are not a topic.

\beginsubsection 3.2. THE TASK SCHEDULER.

We start from the general assumption that, at any given time, a number
of well-determined tasks are ready in the system to be
serviced. Remember that two categories of tasks exist: Interactive
tasks and background tasks. They differ substantially in the criteria
of activation or reactivation and in the priority of
dispatching. Interactive tasks are (re)activated exclusively upon
interactions by the user and are dispatched with high priority. In
contrast, background tasks are polled with low priority.

We already know that interactive tasks are activated by sending messages. The types of messages used for this purpose are {\tt InputMsg\/} and {\tt ControlMsg\/} reporting keyboard events and mouse events respectively. Slightly simplified, they are declared as
\begintt
InputMsg = RECORD (ViewerMsg)
  id: INTEGER;
  X, Y: INTEGER;
  keys: SET;
  ch: CHAR
END;

ControlMsg = RECORD (ViewerMsg)
  id: INTEGER;
  X, Y: INTEGER
END;
\endtt

\noindent The field {\tt id\/} specifies the exact request transmitted with this
specific reactivation. In the case of {\tt InputMsg\/} the possible
requests are consume (the character specified by field {\tt ch\/}) and
track (mouse, starting from state given by keys and {\tt X\/}, {\tt
  Y\/}). In case of {\tt ControlMsg\/} the choice is mark (the viewer
at position {\tt X\/}, {\tt Y\/}) or neutralize. Mark means moving the
global system pointer (typically represented as a star-shaped mark) to
the current position of the mouse. Neutralizing a viewer is equivalent
to removing all marks and graphical attributes from this viewer.

All tasking facilities are collected in one program module, called
{\tt Oberon\/}. In particular, the module's definition exposes the
declarations of the abstract data type Task and of the message types
{\tt InputMsg\/} and {\tt ControlMsg\/}. The module's most important contribution,
however, is the task scheduler (often referred to as ``Oberon loop'')
that can be regarded as the system's dynamic center.

Before studying the scheduler in detail we need some more
preparation. We start with the institution of the {\it focus viewer\/}. By
definition, this is a distinguished viewer that by convention consumes
subsequent keyboard input. Note that we identify the focus viewer with
the focus task, hereby making use of the one-to-one correspondence
between viewers and tasks.

Module Oberon provides the following facilities in connection with the
focus viewer: A global variable {\tt FocusViewer\/}, a procedure {\tt PassFocus\/} for
transferring the role of focus to a new viewer, and a defocus variant
of {\tt ControlMsg\/} for notifying the old focus viewer of such a transfer.

The implementation details of the abstract data type {\tt Task\/} are hidden
from the clients. It is sufficient to know that all task descriptors
are organized in a ring and that a pointer points to the previously
activated task. The ring is guaranteed never to be empty because the
above mentioned garbage collector is installed as a permanent sentinel
task at system loading time.

The following is a slightly abstracted version of the actual scheduler
code operating on the task ring. It should be associated with
procedure {\tt Loop\/} in the module {\tt Oberon\/}.
\begintt
get mouse position and state of keys;
REPEAT
  IF keyboard input available THEN read character
    IF character is escape THEN
      broadcast neutralize message to viewers
    ELSIF character is mark THEN
      send mark message to viewer containing mouse
    ELSE send consume message to focus viewer
    END;
    get mouse position and state of keys
  ELSIF at least one key pressed THEN
    REPEAT
      send track message to viewer containing mouse;
      get mouse position and state of keys
    UNTIL all keys released
  ELSE (*no key pressed*)
    send track message to viewer containing mouse;
    take next task in ring as current task;
    call its handler (if specified time period has elapsed)
    get mouse position and state of keys
  END
UNTIL FALSE
\endtt

\noindent The system executes a sequence of uninterrupted procedures
(tasks). Interactive tasks are triggered by input data being present,
either from the keyboard, the mouse, or other input
sources. Background tasks are taken up in a round-robin
manner. Interactive tasks have priority.

Having consciously excluded exceptional program behavior in our
explanations so far, some comments about the way of runtime
continuation in the case of a failing task or, in other words, in the
case of a {\it trap\/} are in order here. On the (abstract) level of tasks, we
can identify three sequential actions of recovery taken after a
program failure:
\begintt
recovery after program failure =
BEGIN save current system state;
  call installed trap handler;
  roll back to start of task scheduler
END
\endtt

\noindent Essentially, the system state is determined by the values of all
global and local variables at a given time. The trap handler typically
opens an extra viewer displaying the cause of the trap and the saved
system state. Notice in the program fragment above that background
tasks are removed from the ring after failing. This is an effective
precaution against cascades of repeated failures. Obviously, no such
precaution is necessary in the case of interactive tasks because their
reactivation is under control of the user of the system.

Summarizing the essence of the tasking system: Oberon is a
multitasking system based on a two-category model. Interactive tasks
are interfacing with the display system and are scheduled with high
priority upon user interactions. Background tasks are stand-alone and
are scheduled with low priority. Task activations are modeled as
message passing and eventually as calls of procedures assigned to
variables. They are sequentially ordered and controlled by a single
thread of control.

\beginsubsection 3.3. THE CONCEPT OF COMMAND.

An operating system constitutes a general purpose platform on which
application software packages can build upon. To software designers
the platform appears as interface to ``the system'' and (in particular)
to the underlying hardware. Unfortunately, interfaces defined by
conventional operating systems often suffer from an all too primitive
access mechanism that is based solely on the concept of ``software interrupt''
or ``supervisor call'' and on files taking the role of
``connecting pipes''. The situation is especially ironic when compared
with the development of high-level programming languages towards
extreme abstraction.

We have put greatest emphasis in Oberon on closing the {\it semantic gap\/}
between application software packages and the system platform. The
result of our efforts is a highly expressive and consistent
{\it application programming interface\/} (API) in the form of an explicit
hierarchy of module definitions. Perhaps the most significant and most
notable outcome of this approach is a collection of very powerful and
system-wide abstract data types like
{\tt Task\/}, {\tt Frame\/}, {\tt Viewer\/}, {\tt File\/}, {\tt Font\/},
{\tt Text\/}, {\tt Module\/}, {\tt Reader\/}, {\tt Scanner\/}, {\tt Writer\/} etc..

\beginsubsubsection 3.3.1. Atomic actions.

The most important generic function of any operating system is
executing {\it programs\/}. A clarification of the term {\it program\/} as it is used
in Oberon comprises two views: a {\it static\/} one and a {\it dynamic\/}
one. Statically, an Oberon program is simply a package of software
together with an entry point. More formally, an Oberon program is a
pair {\tt (M*, P)\/}, where {\tt M\/} is an arbitrary module, {\tt P\/} is an exported
parameterless procedure of {\tt M\/}, and {\tt M*\/} denotes the hierarchy consisting
of {\tt M\/} itself and of all directly and indirectly imported modules. Note
that two hierarchies {\tt M*\/} and {\tt N*\/} are not generally disjoint, even if {\tt M\/}
and {\tt N\/} are different modules. Rather, their intersection is a superset
of the operating system.

Viewed dynamically, an Oberon program is defined as an atomic action
(often called {\it command\/}) operating on the global system state, where
{\it atomic\/} means ``without user interaction''. This definition is just a
necessary consequence of our model of non-preemptive task scheduling
with the benefit of a single carrier thread. We can argue like this:
When a traditional interactive program requires input from the user,
the current task is normally preempted in favor of another task that
produces the required input data. Therefore, a traditional interactive
program can be viewed as a sequence of atomic actions interrupted by
actions that possibly belong to other programs. Whereas in traditional
systems these interruptions may occur at any time, in Oberon they can
occur only after the completion of a task, of a command.

Quintessentially, Oberon programs are represented in the form of
{\it commands\/} that are in the form of exported parameterless procedures
that do not interact with the user of the system.

Returning to the calling and execution of programs we now arrive at
the following refined code version:
\begintt
call program (M*, P) = BEGIN
  load module hierarchy M*; call command P
END
\endtt

\noindent The system interface to the command mechanism itself is again provided
by module {\tt Oberon\/}. Its primary operation can be paraphrased as
``call a command by its name and pass a list of actual parameters'':
\begintt
PROCEDURE Call (name: ARRAY OF CHAR; par: ParList;
                VAR res: INTEGER);
\endtt

\noindent {\tt name\/} is the name of the desired command in the form {\tt M.P\/}, {\tt par\/} is the
list of actual parameters, and {\tt res\/} is a result code. But in fact we
have separated the setting of parameters from the actual
call. Parameters are set by calling
\begintt
PROCEDURE SetPar (F: Display.Frame; T: Texts.Text;
                  pos: INTEGER);
\endtt

\noindent and the actual call is achieved by calling
\begintt
PROCEDURE Call (name: ARRAY OF CHAR; VAR res: INTEGER);
\endtt

\noindent The pair {\tt (T, pos)\/} specifies the starting position of a textual
parameter list. {\tt F\/} indicates the calling viewer. Notice the occurrence
of yet another abstract data type of name {\tt Text\/} that is exported by
module {\tt Texts\/}. We shall devote Chapter 5 to a thorough discussion of
Oberon's text system. For the moment we can simply look at a text as a
sequence of characters.

The list of actual parameters is handed over to the called command by
module {\tt Oberon\/} in the form of an exported global variable {\tt Par\/}:
\begintt
Par: RECORD vwr: Viewers.Viewer;
       frame: Display.Frame;
       text: Texts.Text;
       pos: INTEGER
     END
\endtt
In principle, commands operate on the entire system and can access the
current global state via the system's powerful abstract modular
interface, of which the list of actual parameters is just one
component. Another one is the so-called {\it system log\/} which is a
system-wide protocol reporting on the progress of command execution
and on exceptional events in chronological order. The log is
represented as a global variable of type {\tt Text\/}:
\begintt
Log: Texts.Text;
\endtt

\noindent It should have become clear by now that implementers of commands may
rely on a rich arsenal of abstract global facilities that reflect the
current system state and make it accessible. In other words, they may
rely on a high degree of system integration. Therefore, Oberon
features an extraordinarily broad spectrum of mutually integrated
facilities. For example, the system distinguishes itself by a complete
integration of the abstract data types {\tt Viewer\/} and {\tt Text\/} that we
encountered above. They will be the subject of Chapters 4 and 5.

Module {\tt Oberon\/} assists the integration of these types with the
following conceptual features, of which the first two are familiar to
us already: Standard parameter list for commands, system log, generic
text selection, and generic copy viewer. At this point we should add a
word of clarification to our use of the term ``generic''. It is
synonymous with ``interpretable individually by any viewer (interactive task)''
and is typically used in connection with messages or orders
whose receiver's exact identity is unknown.

Let us now go into a brief discussion of the generic facilities
without, however, leaving the level of our current abstraction and
understanding.

\beginsubsubsection 3.3.2. Generic text selection.

Textual selections are characterized by a text, a stretch of
characters within that text, and a time stamp. Without further
qualification ``the text selection'' always means ``the most recent text selection''.
It can be obtained programmatically by calling procedure GetSelection:
\begintt
PROCEDURE GetSelection (VAR text: Texts.Text;
                        VAR beg, end, time: LONGINT);
\endtt

\noindent The parameters specify the desired stretch of text starting
at position beg and ending at end - 1 as well as the associated time
stamp. The procedure is implemented in form of a broadcast of a so-
called selection message to all viewers. The declaration of this
message is
\begintt
SelectionMsg = RECORD (ViewerMsg)
  time: INTEGER;
  text: Texts.Text;
  beg, end: INTEGER
END;
\endtt

\beginsubsubsection 3.3.3. Generic copy viewer.

Generic copying is synonymous with reproducing and cloning. It is the
most elementary generic operation possible. Again, a variant of type
{\tt ViewerMsg\/} is used for the purpose of transmitting requests of the
desired type:
\begintt
CopyMsg = RECORD (ViewerMsg) vwr: Viewers.Viewer END
\endtt 

\noindent Receivers of a {\tt copy\/} message typically generate a clone of
themselves and return it to the sender via field {\tt vwr\/}.

Let us now summarize this Section:. Oberon is an operating system that
presents itself to its clients in the form of a highly expressive
abstract modular interface that exports many powerful abstract data
types like, for example, {\tt Viewer\/} and {\tt Text\/}. A rich arsenal of global
data types and generic facilities serve the purpose of system
integration at a high degree. Programs in Oberon are modeled as
so-called commands, i.e. as exported parameterless procedures that do
not interact with the user. The collection of commands provided by a
module appears as its user interface. Parameters are passed to
commands via a global parameter list, registered by the calling task
in the central module {\tt Oberon\/}. Commands operate on the global state of
the system.

\beginsubsection 3.4. TOOLBOXES.

Modules typically appear in three different forms. The first is a
module that encapsulates some data, letting them be accessed only
through exported procedures and functions. A good example is Module
{\tt FileDir\/}, encapsulating the file directory and protecting it from
disruptive access. A second kind is the module representing an
{\it abstract data type\/}, exporting a type and its associated
operators. Typical examples are modules {\tt Files\/}, {\tt Modules\/}, {\tt Viewers\/}, and
{\tt Texts\/}. A third kind is the collection of procedures pertaining to the
same topic, such as module {\tt RS-232\/} handling communication over a serial
line.

Oberon adds a fourth form: the {\it toolbox\/}. By definition, this is a pure
collection of commands in the sense of the previous section. Toolboxes
distinguish themselves principally from the other forms of modules by
the fact that they lie on top of the modular hierarchy. Toolbox
modules are ``imported'' by system users at run-time. In other words,
their definitions define the user interface. Typical examples are
modules {\tt System\/} and {\tt Edit\/}. As a rule of thumb there exists a toolbox for
every topic or application.

As an example of a toolbox definition we quote an annotated version of
module {\tt System\/}:
\begintt
DEFINITION System;

(*System management, Chapters 3 and 8*)
  PROCEDURE SetUser; (*identification*)
  PROCEDURE SetFont; (*for typed text*)
  PROCEDURE SetColor; (*for typed text and graphics*)
  PROCEDURE SetOffset; (*for typed text*)
  PROCEDURE Date; (*set or display time and date*)
  PROCEDURE Collect; (*garbage*)
  
(*Display management, Chapter 4*)
  PROCEDURE Open; (*viewer*)
  PROCEDURE Close; (*viewer*)
  PROCEDURE CloseTrack;
  PROCEDURE Recall; (*most recently closed viewer*)
  PROCEDURE Copy; (*viewer*)
  PROCEDURE Grow; (*viewer*)
  PROCEDURE Clear; (*clear log*)
  
(*Module management, Chapter 6*)
  PROCEDURE Free; (*specified modules*)
  PROCEDURE ShowCommands; (*of specified module*)
  PROCEDURE ShowModules; (*list loaded modules*)
  
(*File management, Chapter 7*)
  PROCEDURE Directory;
  PROCEDURE CopyFiles;
  PROCEDURE RenameFiles;
  PROCEDURE DeleteFiles;)
  
(*System inspection, Chapter 8*)
  PROCEDURE Watch; (*tasks, memory and disk storage*)
END System;
\endtt

\noindent An important consequence of our integrated systems approach is the
possibility of constructing a universal, interactive 
{\it command interpreter\/} bound to viewers of {\it textual contents\/}. If the text obeys
the following syntax (specified in Extended Backus-Naur Form EBNF), we
call it {\it command tool\/}:
\begintt
CommandTool = { [Comment] CommandName [ParameterList] }.
\endtt

\noindent If present, the parameter list is made available to the called command
via fields text and pos in the global variable Par that is exported
from module {\tt Oberon\/}. Because this parameter list is interpreted
individually by each command, its format is completely open. However,
we postulate some conventions and rules for the purpose of a
standardized user interface:

1.) The elements of a textual parameter list are universal syntactical tokens like name, literal string, integer, real number, and special character.

2.) An arrow ``{\tt \^\/}'' in the textual parameter list refers to the current text selection for continuation. In the special case of the arrow following the command name immediately, the entire parameter list is represented by the text selection.

3.) An asterisk ``{\tt *}'' in the textual parameter list refers to the currently marked viewer. Typically, the asterisk replaces the name of a file. In such a case the contents of the viewer marked by the system pointer (star) is processed by the command interpreter instead of the contents of a file.

4.) An at-character ``{\tt @}' in the textual parameter list indicates that the selection marks the (beginning of the) text which is taken as operand.

5.) A terminator-character ``{\tt \~{}}'' terminates the textual parameter list in case of a variable number of parameters.

Because command tools are ordinary, editable texts (in contrast to
menus in conventional systems) they can be customized ``on the fly'',
which makes the system highly flexible. We refer again to Figure 3.1
that shows a typical Oberon screen layout consisting of two vertical
tracks, a wider user track on the left and a narrow system track on
the right. Three documents are displayed in the user track: A text, a
graphic, and a picture. In the system track we find one log-viewer
displaying the system log, two tool-viewers making available the
standard system tool and a customized private tool respectively.

In concluding this Chapter, let us exemplify the concepts of command
and tool by the system control section of the {\tt System\/}
toolbox. Consisting of the commands {\tt SetUser\/}, {\tt Date\/},
{\tt SetFont\/}, {\tt SetColor\/}, and {\tt Collect\/} it is used to control
system-wide facilities. In detail, their function is installing the
user's identification, displaying or setting the system date and time,
presetting the system type-font for typed text, setting the system
color, and activating the garbage collector.

In summary, a toolbox is a special form of an {\tt Oberon\/} module. It is
defined as a collection of commands. Appearing at the top of the
modular hierarchy the toolboxes in their entirety fix the system’s
user interface. Command tools are sequences of textually represented
command calls. They are editable and customizable. In a typical Oberon
screen layout the tools are displayed in viewers within the system
track.

\beginchapter CHAPTER 4: THE DISPLAY SYSTEM.

The display screen is the most important part of the interface
presented by a personal workstation to its users. At first sight, it
simply represents a rectangular output area. However, in combination
with the mouse, it quickly develops into a sophisticated interactive
input/output platform of almost unlimited flexibility. It is mainly
its Janus-faced characteristic that makes the display screen stand out
from ordinary external devices to be managed by the operating
system. In the current chapter we shall give more detailed insight
into the reasons for the central position the display system takes
within the operating system, and for its determining influence on the
entire system architecture. In particular, we shall show that the
display system is a natural basis or anchor for functional
extensibility.

\beginsubsection 4.1. THE SCREEN LAYOUT MODEL.

In the early seventies, Xerox PARC in California launched the
Smalltalk-project with the goal of conceiving and developing new and
more natural ways to communicate with personal computers
\footnote1{[Goldberg] A. Goldberg, Smalltalk-80: The Interactive Programming Environment, Addison-Wesley 1984.}. Perhaps the most conspicuous among several significant
achievements of this endeavor is the idea of applying the desktop
metaphor to the display screen. This metaphor comprises a desktop and
a collection of possibly mutually overlapping pages of paper that are
laid out on the desktop. By projecting such a configuration onto the
surface of a screen we get the familiar picture of Figure 4.1 showing
a collection of partially or totally visible rectangular areas on a
background, so-called {\it windows\/} or {\it viewers\/}.

TODO: Figure 4.1 Desktop showing partially overlapping viewers

The desktop metaphor is used by many modern operating systems and user
interface shells both as a natural model for the system to separate
displayed data belonging to different tasks, and as a powerful tool
for users to organize the display screen interactively, according to
individual taste and preference. However, there are inherent drawbacks
in the metaphor. They are primarily connected with
overlapping. Firstly, any efficient management of overlapping viewers
must rely on a subordinate management of (arbitrary) sub-rectangles
and on sophisticated clipping operations. This is so because partially
overlapped viewers must be partially restored under control of the
viewer manager. For example, in Figure 4.1, rectangles a, b, and c in
viewer B ought to be restored individually after closing of viewer
A. Secondly, there is a significant danger of covering viewers
completely and losing them forever. And thirdly, no canonical
heuristic algorithms exist for automatic allocation of screen space to
newly opened viewers.

Experience has shown that partial overlapping is desirable and
beneficial in rare cases only, and so the additional complexity of its
management \footnote1{C. Binding, User Interface Components based on a Multiple Window Package, University of Washington, Seattle, Technical Report 85-08-07.}
\footnote2{M. Wille, Overview: Entwurf und Realisierung eines Fenstersystems für Arbeitsplatzrechner, Diss. ETH Nr. 8771, 1988.}
is hard to justify. Therefore, alternate
strategies to structure a display screen have been looked for. An
interesting class of established solutions can be titled as {\it tiling\/}.
There are several variants of tiling \footnote3{[Cohen] E.S. Cohen, E.T. Smith, L.A. Iverson, Constraint-Based Tiled Windows, IEEE, 1985}. Perhaps the
most obvious one (because the most unconstrained one) is based on
iterated horizontal or vertical splitting of existing
viewers. Starting with the full screen and successively opening
viewers A, B, C, D, E, and F we get to a configuration as in Figure
4.2.

TODO: Figure 4.2 Viewer configuration resulting from unconstrained tiling

A second variant is hierarchic tiling. Again, the hierarchy starts
with a full screen that is now decomposed into a number of vertical
tracks, each of which is further decomposed into a number of
horizontal viewers. We decided in favor of this kind of tiling in
Oberon, mainly because the algorithm of reusing the area of a closed
viewer is simpler and more uniform. For example, assume that in Figure
4.2 viewer F has been closed. Then, it is straightforward to reverse
the previous opening operation by extending viewer E at its bottom
end. However, if the closed viewer is B, no such simple procedure
exists. For example, the freed area can be shared between viewers C
and D by making them extend to their left. Clearly, no such
complicated situations can occur in the case of hierarchic tiling.

Hierarchic tiling is also used in Xerox PARC's Cedar system
\footnote4{[Teitelman] W. Teitelman, "A tour through Cedar", IEEE Software, 1, (2), 44-73 (1984).
}. However, the Oberon variant differs from the Cedar
variant in some respects. Firstly, Oberon supports quick temporary
context switching by overlaying one track or any contiguous sequence
of tracks with new layers. In Figure 4.3 a snapshot of a standard
Oberon display screen is graphically represented. It suggests two
original tracks and two levels of overlay, where the top layer is
screen-filling. Secondly, unlike Cedar display screens, Oberon
displays do not provide reserved areas for system-wide facilities,
Standard Cedar screens feature a command row at the top and an icon
row at the bottom. And thirdly, Oberon is based on a different
heuristic strategy for the automatic placement of new viewers. As a
Cedar default invariant, the area of every track is divided up evenly
among the viewers in this track. When a new viewer is to be placed,
the existing viewers in the track are requested to reduce their size
and move up appropriately. The newly opened viewer is then allocated
in the freed spot at the bottom. In contrast, Oberon normally splits
the largest existing viewer in a given track into two halves of equal
size. As an advantage of this latter allocation strategy we note that
existing contents are kept stable.

TODO: Figure 4.3 Overlay of tracks and sequences of tracks

\beginsubsection 4.2. VIEWERS AS OBJECTS.

Although everybody seems to agree on the meaning of the term viewer,
no two different system designers actually do. The original role of a
viewer as merely a separate display area has meanwhile become heavily
overloaded with additional functionality. Depending on the underlying
system are viewers' individual views on a certain configuration of
objects, carriers of tasks, processes, applications, etc. Therefore,
we first need to define our own precise understanding of the concept
of viewer.

The best guide to this aim is the abstract data type {\tt Viewer\/} that we
introduced in Chapter 3. We recapitulate: Type {\tt Viewer\/} serves as a
template describing viewers abstractly as ``black boxes'' in terms of a
state of visibility, a rectangle on the display screen, and a message
handler. The exact functional interface provided by a given variant of
viewer is determined by the set of messages accepted. This set is
structured as a customized hierarchy of type extensions.

We can now obtain a more concrete specification of the role of viewer
by identifying some basic categories of universal messages that are
expected to be accepted by all variants of viewer. For example, we
know that messages reporting about user interactions as well as
messages defining a generic operation are universal. These two
categories of universal messages document the roles of viewers as
interactive tasks and as parts of an integrated system respectively.

In total, there are four such categories. They are here listed
together with the corresponding topics and message dispatchers:

TODO: Table
Dispatcher
Task scheduler Command interpreter Viewer manager Document manager
Topic
dispatching of task processing of command organizing display area operating on document
Message
reports user interaction defines generic operation change of location or size change of contents or format

These topics essentially define the role of Oberon viewers. In short,
we may look at an Oberon viewer as a non-overlapped rectangular box on
the screen both acting as an integrated display area for some objects
of a document and representing an interactive task in the form of a
sensitive editing area.

Shifting emphasis a little and regarding the various message
dispatchers as subsystems, we recognize immediately the role of
viewers as integrators of the different subsystems via message-based
interfaces. In this light type Viewer appears as a common
object-oriented basis of Oberon's subsystems.

The topics listed above constitute some kind of backbone of the
contents of the Chapters 3, 4 and 5. Task scheduling and command
interpreting are already familiar to us from Sections 3.2 and
3.3. Viewer management and text management will be the topics of
Sections 4.4 and 5.2 respectively. Thereby, the built-in type Text
will serve as a prime example of a document type.

The activities that a viewer performs are basically controlled by
events or, more precisely, by messages representing {\it event notices\/}. We
shall explain this in detail in Sections 4.4 and 5.3 in the cases of
an abstract class of standard viewers and a class of viewers
displaying standard text respectively.

Here is a preliminary overview of some archetypal kinds of message:

TODO: List bullets
• After each key stroke a keyboard message containing the typed character is sent to the current focus viewer and after each mouse click a mouse message reporting the new state of the mouse is sent to the viewer containing the current mouse position.

• A message often represents some generic operation that is expected to be interpreted individually by its recipients. Obvious examples in our context are ``return current textual selection'', ``copy-over stretch of text'', and ``produce a copy (clone)''. Notice that generic operations are the key to extensibility.

• In a tiling viewer environment, every opening of a new viewer and every change of size or location of an existing viewer has an obvious effect on adjacent viewers. The viewer manager therefore issues a message for every affected viewer requesting it to adjust its size appropriately.

• Whenever the contents or the format of a document has changed, a message notifying all visible viewers of the change is broadcast. Notice that broadcasting messages by a model (document) to the entirety of its potential views (viewers) is an interesting implementation of the famous MVC (model-view-controller) pattern that dispenses models from ``knowing'' (registering) their views.

\beginsubsection 4.3. FRAMES AS BASIC DISPLAY ENTITIES.

When we introduced viewers in Chapter 3 and in the previous section,
we simplified with the aim of abstraction. We know already that
viewers appear as elements of second order in the tiling
hierarchy. Having treated them as black boxes so far we have not
revealed anything about the continuation of the hierarchy. As a matter
of fact, viewers are neither elementary display entities nor
atoms. They are just a special case of so-called {\it display frames\/}. Display frames or frames in short are arbitrary rectangles
displaying a collection of objects or an excerpt of a document. In
particular, frames may recursively contain other frames, a capability
that makes them an extremely powerful tool for any display organizer.

The type {\tt Frame\/} is declared as
\begintt
Frame = POINTER TO FrameDesc;
FrameDesc = RECORD
  next, dsc: Frame;
  X, Y, W, H: INTEGER;
  handle: Handler
END;
\endtt

\noindent The components {\tt next\/} and {\tt dsc\/} are connections to further frames. Their
names suggest a multi-level recursive hierarchical structure: {\tt next\/}
points to the next frame on the same level, while {\tt dsc\/} points to the
(first) descendant, i.e. to the next lower level of the hierarchy of
nested frames. {\tt X\/}, {\tt Y\/}, {\tt W\/}, {\tt H\/}, and the handler handle serve the original
purpose to that we introduced them. In particular, the handler allows
frames to react individually on the receipt of messages. Its type is
\begintt
Handler = PROCEDURE (F: Frame; VAR M: FrameMsg);
\endtt

\noindent where FrameMsg represents the root of a potentially unlimited tree hierarchy of possible messages to frames:
\begintt
FrameMsg = RECORD END;
\endtt

\noindent Having now introduced the concept of frames, we can reveal the whole truth about viewers. As a matter of fact, type {\tt Viewer\/} is a derived type, it is a type extension of {\tt Frame\/}:
\begintt
Viewer = POINTER TO ViewerDesc;
ViewerDesc = RECORD (FrameDesc)
  state: INTEGER
END;
\endtt

\noindent These declarations formally express the fact that viewers
are nothing but a special case (or variant or subclass) of general
frames, additionally featuring a state of visibility. In particular,
viewers inherit the hierarchical structure of frames. This is an
extremely useful property immediately opening an unlimited spectrum of
possibilities for designers of a specific subclass of viewers to
organize the representing rectangular area. For example, the area of
viewers of, say, class {\tt Desktop\/} may take the role of a background being
covered by an arbitrary collection of possibly mutually overlapping
frames. In other words, our decision of using a tiling viewer scheme
{\it globally\/} can easily be overwritten {\it locally\/}.

An even more important example of a predefined structure is provided
by the abstract class of so-called {\it menu viewers\/} whose shape is
familiar from most snapshots taken of the standard Oberon display
screen. A menu viewer consists of a thin rectangular boundary line and
an interior area being vertically decomposed into a menu region at the
top and a contents region at the bottom (see Figure 4.4).

TODO: Figure 4.4 The compositional structure of a menu viewer

In terms of data structures, the class of menu viewers is defined as a
type extension of {\tt Viewer\/} with an additional component {\tt menuH\/}
specifying the height of the menu frame:
\begintt
MenuViewer = POINTER TO MenuViewerDesc;
MenuViewerDesc = RECORD (ViewerDesc)
  menuH: INTEGER
END;
\endtt

\noindent Each menu viewer {\tt V\/} specifies exactly two
descendants: The menu frame {\tt V.dsc\/} and the frame of main
contents or main frame {\tt V.dsc.next\/}. Absolutely nothing is fixed
about the contents of the two descendant frames. In the standard case,
however, the menu frame is a text frame, displaying a line of commands
in inverse video mode. By definition, the nature of the main frame
specifies the type of the viewer. If it is a text frame as well, then
we call the viewer a {\it text viewer\/}, if it a graphics frame, we call it a
{\it graphics viewer\/} etc.

\beginsubsection 4.4. DISPLAY MANAGEMENT.

Oberon's display system comprises two main topics: Viewer management
and cursor handling. Let us first turn to the much more involved topic
of viewer management and postpone cursor handling to the end of this
Section. Before we can actually begin our explanations we need to
introduce the concept of the {\it logical display area\/}. It is modeled as a
two-dimensional Cartesian plane housing the totality of objects to be
displayed. The essential point of this abstraction is a rigorous
decoupling of any aspects of physical display devices. As a matter of
fact, any concrete assignment of display monitors to certain finite
regions of the display area is a pure matter of configuring the
system.

Being a subsystem of a system with a well-defined modular structure
the display system appears in the form of a small hierarchy of
modules. Its core is a linearly ordered set consisting of three
modules: {\tt Display\/}, {\tt Viewers\/}, and {\tt MenuViewers\/}, the latter building upon
the formers. Conceptually, each module contributes an associated class
of display-oriented objects and a collection of related service
routines.

The following is an overview of the subsystem viewer
management. Modules on upper lines import modules on lower lines and
types on upper lines extend types on lower lines.

TODO: table
Module
MenuViewer Viewers Display
Type
Viewer Viewer Frame
Service
Message handling for menu viewers Tiling viewer management Block-oriented raster operations

Inspecting the column titled Type we recognize precisely our familiar
types {\tt Frame\/}, {\tt Viewer\/}, and {\tt MenuViewer\/} respectively, where the latter is
an abbreviation of {\tt MenuViewers.Viewer\/}.

In addition to the core modules of the display system a section in
module {\tt Oberon\/} provides a specialized application programming interface
(API) that simplifies the use of the viewer management package by
applications in the case of standard Oberon display configurations. We
shall come back to this topic in Section 4.6.

For the moment let us concentrate on the core of the viewer management
and in particular on the modules {\tt Viewers\/} and {\tt MenuViewers\/}, saving the
discussion of the module {\tt Display\/} for the next section. Typically, we
start the presentation of a module by listing and commenting its
definition, and we refer to subsequent listings for its
implementation.

\beginsubsubsection 4.4.1. Viewers.


Focusing first on module {\tt Viewers\/} we can roughly define the domain of
its responsibility as ``initializing and maintaining the global layout
of the display area''. From the previous discussion we are well
acquainted already with the structure of the global display space as
well as with its building blocks: The display area is hierarchically
tiled with display frames, where the first two levels in the frame
hierarchy correspond to {\it tracks\/} and {\it viewers\/} respectively.

This is the formal definition:
\begintt
DEFINITION Viewers;
  IMPORT Display;
  CONST restore = 0; modify = 1; suspend = 2; (*message ids*)
  TYPE Viewer = POINTER TO ViewerDesc;
       ViewerDesc = RECORD (Display.FrameDesc)
         state: INTEGER
       END;
       ViewerMsg = RECORD (Display.FrameMsg)
         id: INTEGER;
         X, Y, W, H: INTEGER;
         state: INTEGER
       END;

  VAR curW: INTEGER;
       
  (*track handling*)
  PROCEDURE InitTrack (W, H: INTEGER; Filler: Viewer);
  PROCEDURE OpenTrack (X, W: INTEGER; Filler: Viewer);
  PROCEDURE CloseTrack (X: INTEGER);

  (*viewer handling*)
  PROCEDURE Open (V: Viewer; X, Y: INTEGER);
  PROCEDURE Change (V: Viewer; Y: INTEGER);
  PROCEDURE Close (V: Viewer);

  (*miscellaneous*)
  PROCEDURE This (X, Y: INTEGER): Viewer;
  PROCEDURE Next (V: Viewer): Viewer;
  PROCEDURE Recall (VAR V: Viewer);
  PROCEDURE Locate (X, H: INTEGER; VAR fil, bot, alt, max: Viewer);
  PROCEDURE Broadcast (VAR M: Display.FrameMsg);
END Viewers.
\endtt

\noindent Some comments: A first group of procedures consisting of {\tt
InitTrack\/}, {\tt OpenTrack\/}, and {\tt CloseTrack\/} supports the
track structure of the display area. {\tt InitTrack\/} creates a new track of
width {\tt W\/} and height {\tt H\/} by partitioning off a vertical strip of width {\tt W\/}
from the display area. In addition, {\tt InitTrack\/} initializes the newly
created track with a filler viewer that is supplied as a
parameter. The filler viewer essentially serves as background filling
up the track at its top end. It reduces to height 0 if the track is
covered completely by productive viewers.

Configuring the display area is part of system initialization after
startup. It amounts to executing a sequence of steps of the form
\begintt
NEW(Filler); Filler.handle := HandleFiller; InitTrack(W, H, Filler)
\endtt

\noindent where {\tt HandleFiller\/} is supposed to handle messages
that require modifications of size and cursor drawing.

The global variable {\tt curW\/} indicates the width of the already configured
part of the display area. Note that configuring starts with {\tt x = 0\/} and
is non-reversible in the sense that the grid defined by the
initialized tracks cannot be refined later. However, remember that it
can be coarsened at any time by overlaying a contiguous sequence of
existing tracks by a single new track.

Procedure {\tt OpenTrack\/} serves exactly this purpose. The track (or
sequence of tracks) to be overlaid in the display-area must be spanned
by the segment {\tt [X, X + W)\/}. Procedure {\tt CloseTrack\/} is inverse to
{\tt OpenTrack\/}. It is called to close the (topmost) track located at {\tt X\/} in
the display area, and to restore the previously covered track (or
sequence of tracks).

The next three procedures are used to organize viewers within
individual tracks. Procedure {\tt Open\/} allocates a given viewer at a given
position. More precisely, {\tt Open\/} locates the viewer containing the point
{\tt (X, Y)\/}, splits it horizontally at height {\tt Y\/}, and opens the viewer {\tt V\/} in
the lower part of the area. In the special case of {\tt Y\/} coinciding with
the upper boundary line of the located viewer this is closed
automatically. Procedure {\tt Change\/} allows to change the height of a given
viewer {\tt V\/} by moving its upper boundary line to a new location {\tt Y\/} (within
the limits of its neighbors). Procedure {\tt Close\/} removes the given viewer
{\tt V\/} from the display area. Figure 4.5 makes these operations clear.

TODO: Figure 4.5 Basic operations on viewers

The last group of procedures provides miscellaneous
services. Procedure {\tt This\/} identifies the viewer displayed at {\tt (X, Y)\/}.
Procedure {\tt Next\/} returns the next upper neighbor of a given
displayed viewer {\tt V\/}. Procedure {\tt Recall\/} allows recalling and restoring
the most recently closed viewer. {\tt Locate\/} is a procedure that assists
heuristic allocation of new viewers. For any given track and desired
minimum height, procedure {\tt Locate\/} offers a choice of some distinguished
viewers in the track: the filler viewer, the viewer at the bottom, an
alternative choice, and the viewer of maximum height. Finally,
procedure {\tt Broadcast\/} broadcasts a message to the display area, that is,
sends the given message to all viewers that are currently displayed.

It is now a good time to throw a glance behind the scenes. Let us
start with revealing module {\tt Viewer\/}’s internal data structure. Remember
that according to the principle of information hiding an internal data
structure is fully private to the containing module and accessible
through the module’s procedural interface only. Figure 4.6 shows a
data structure view of the display snapshot taken in Figure 4.4. Note
that the overlaid tracks and viewers are still part of the internal
data structure.

In the data structure we recognize an anchor that represents the
display area and points to a list of tracks, each of them in turn
pointing to a list of viewers, each of them in turn pointing to a list
of arbitrary sub-frames. Both the list of tracks and the list of
viewers are closed to a ring, where the filler track (filling up the
display area) and the filler viewers (filling up the tracks) act as
anchors. Additionally, each track points to a (possibly empty) list of
tracks lying underneath. These frames are invisible on the display,
and shaded in Figure 4.6.

TODO: Figure 4.6 A snapshot of the internal data structure corresponding to Figure 4.3

Technically, the track descriptor type {\tt TrackDesc\/} is a private
extension of the viewer descriptor type {\tt ViewerDesc\/}. Repeating the
declarations of viewer descriptors and frame descriptors, we get to
this hierarchy of types:
\begintt
TrackDesc = RECORD (ViewerDesc)
  under: Display.Frame
END;

ViewerDesc = RECORD (FrameDesc)
  state: INTEGER
END;

FrameDesc = RECORD
  next, dsc: Frame;
  X, Y, W, H: INTEGER;
  handle: Handler
END;
\endtt

\noindent It is noteworthy that the data structure of the viewer
manager is heterogeneous with {\tt Frame\/} as base type. It provides
a nice example of a nested hierarchy of frames with the additional
property that the first two levels correspond to the first two levels
in the type hierarchy defined by {\tt Track\/}, {\tt Viewer\/}, and {\tt Frame\/}.


In an object-oriented environment objects are autonomous entities in
principle. However, they may be bound to some higher instance (other
than the system) temporarily. For example, we can look at the objects
belonging to a module's private data structure as bound to this
module. Deciding if an object is currently bound is then a fundamental
problem. In the case of viewers, this information is contained in an
extra instance variable called {\tt state\/}.

As a system invariant, we have for every viewer {\tt V\/}

{\tt V\/} is bound to module {\tt Viewers\/} \Leftrightarrow {\tt V.state} \neq 0

If we call visible any displayed viewer and suspended any viewer that
is covered by an overlaying track we can refine this invariant to

({\tt V\/} is visible \Leftrightarrow ${\tt V.state\/} > 0$) and ( {\tt V\/} is suspended \Leftrightarrow ${\tt V.state\/} < 0$)

TODO: Clean this math shit up

In addition, more detailed information about the kind of viewer {\tt V\/} is given by the magnitude
TODO: absolute value ${\tt V.state\/}$ :

TODO: table
V.state kind of viewer
0 closed
1 filler
-1 productive

The magnitude $ \vert V.state \vert $ is kept invariant by module {\tt Viewers\/}. It could
be used, for example, to distinguish different levels of importance or
preference with the aim of supporting a smarter algorithm for
heuristic allocation of new viewers. The variable state is treated as
read-only by every module other than {\tt Viewers\/}.

We are now sufficiently prepared to understand how the exported
procedures of module {\tt Viewers\/} work behind the scenes. All of them
operate on the internal dynamic data structure just explained. Some
use the structure as a reference only or operate on individual
elements (procedures {\tt This\/}, {\tt Next\/}, {\tt Locate\/}, {\tt Change\/}), others add new
elements to the structure (procedures {\tt InitTrack\/}, {\tt OpenTrack\/}, {\tt Open\/}), and
even others remove elements (procedures {\tt CloseTrack\/}, {\tt Close\/}). Most
procedures have side-effects on the size or state of existing
elements.

Let us now change perspective and look at module {\tt Viewers\/} as a general
low-level manager of viewers whose exact contents are unknown to it
(and whose controlling software might have been developed years
later). In short, let us look at module {\tt Viewers\/} as a manager of black
boxes. Such an abstraction immediately makes it impossible for the
implementation to call fixed procedures for, say, changing a viewer's
size or state. The facility needed is a message-oriented interface.
\begintt
TYPE ViewerMsg = RECORD (Display.FrameMsg)
  id: INTEGER;
  X, Y, W, H: INTEGER;
  state: INTEGER
END;
\endtt

\noindent There exist three variants of {\tt Viewer\/} messages, discriminated by the
field {\tt id\/}: Restore contents, modify height (extend or reduce at
bottom), and suspend (close temporarily or permanently). The
additional components of the message inform about the desired new
location, size, and state.

The following table lists senders, messages, and recipients of viewer messages.

TODO: table
Originator
OpenTrack CloseTrack Open Change Close
Message
Suspend temporarily Suspend permanently Modify or suspend Modify
Suspend permanently
Recipients
Viewers covered by opening track Viewers in closing track
Upper neighbor of opening viewer Upper neighbor of changing viewer Closing viewer

\beginsubsubsection 4.4.2. Menu Viewers.

So far, we have treated viewers abstractly as black boxes. Our next
step is now to focus on a special class of viewers called {\it menu viewers\/}. Remembering the definition given earlier we know that a menu
viewer is characterized by a structure consisting of two vertically
tiled ``descendant'' frames, a {\it menu frame\/} at the top and a {\it frame of contents\/} at the bottom. Because the nature and contents of these
frames are typically unknown by their ``ancestor'' (or ``parent'') viewer,
a collection of abstract messages is again a postulating form of
interface. As net effect, the handling of menu viewers boils down to a
combination of preprocessing, transforming and forwarding messages to
the descendant frames. In short, the display space in Oberon is
hierarchically organized and message passing within the display space
obeys the pattern of strict {\it parental control\/}.

Again, we start our more detailed discussion with a module interface definition:
\begintt
DEFINITION MenuViewers;
  IMPORT Viewers, Display;
  CONST extend = 0; reduce = 1; move = 2; (*message ids*)

  TYPE
    Viewer = POINTER TO ViewerDesc;
    ViewerDesc = RECORD (Viewers.ViewerDesc)
      menuH: INTEGER
    END;
    ModifyMsg = RECORD (Display.FrameMsg)
      id: INTEGER;
      dY, Y, H: INTEGER
   END;
   PROCEDURE Handle (V: Display.Frame; VAR M: Display.FrameMsg);
   PROCEDURE New (Menu, Main: Display.Frame;
                  menuH, X, Y: INTEGER): Viewer;
END MenuViewers.
\endtt

The interface represented by this definition is conspicuously
narrow. There are just two procedures: A generator procedure {\tt New\/} and a
standard message handler {\tt Handle\/}. The generator returns a newly created
menu viewer displaying the two (arbitrary) frames passed as
parameters. The message handler implements the entire ``behavior'' of an
object and in particular the above mentioned message dispatching
functionality.

Message handlers in Oberon are implemented in the form of procedure
variables that obviously must be initialized properly at object
creation time. In other words, some concrete behavior must explicitly
be bound to each object, where different instances of the same object
type could potentially have a different behavior and/or the same
instance could change its behavior during its lifetime. Our object
model is therefore instance-centered.

Conceptually, the creation of an object is an atomic action consisting
of three basic steps:

allocate memory block; install message handler; initialize state variables

\noindent In the case of a standard menu viewer {\tt V\/} this can be expressed as
\begintt
NEW(V);
V.handle := Handle;
V.dsc := Menu;
V.dsc.next := Main;
V.menuH := menuH
\endtt

\noindent With that, calling {\tt New\/} is equivalent with
\begintt
create V; open V at X, Y
\endtt

\noindent where opening {\tt V\/} needs assistance by module {\tt Viewers\/}.

The implementation of procedure {\tt Handle\/} embodies the standard strategy of message handling by
menu viewers. The following code is a coarse-grained view of it.
\begintt
Message handler for menu viewers

IF message reports about user interaction THEN
  IF variant is mouse tracking THEN
    IF mouse is in menu region THEN
      IF mouse is in upper menu region and left key is pressed THEN
        handle changing of viewer
      ELSE delegate handling to menu-frame
      END
    ELSE
      IF mouse is in main-frame THEN delegate handling to main-frame END
    END
  ELSIF variant is keyboard input THEN
    delegate handling to menu frame;
    delegate handling to main frame
  END
ELSIF message defines generic operation THEN
  IF message requests copy (clone) THEN
    send copy message to menu frame to get a copy (clone);
    send copy message to main frame to get a copy (clone);
    create menu viewer clone from copies
  ELSE
    delegate handling to menu frame; delegate handling to main frame
  END
ELSIF message reports about change of contents THEN
  delegate handling to menu frame;
  delegate handling to main frame
ELSIF message requests change of location or size THEN
  IF operation is restore THEN
    draw viewer area and border;
    send modify message to menu frame to make it extend from height 0;
    send modify message to main frame to make it extend from height 0
  ELSIF operation is modify THEN
    IF operation is extend THEN
      extend viewer area and border;
      send modify message to menu frame to make it extend;
      send modify message to main frame to make it extend
    ELSE (*reduce*)
      send modify message to main frame to make it reduce;
      send modify message to menu frame to make it reduce;
      reduce viewer area and border
    END
  ELSIF operation is suspend THEN
    send modify message to main frame to make it reduce to height 0;
    send modify message to menu frame to make it reduce to height 0
  END
END
\endtt

\noindent In principle, the handler acts as a {\it message dispatcher\/} that
either processes a message directly and/or delegates its processing to
the descendant frames. Note that the handler's main alternative
statement discriminates precisely among the four basic categories of
messages.

From the above outlined algorithm handling {\it copy messages\/}, that
is, requests for generating a {\it copy\/} or {\it clone\/} of a menu
viewer, we can derive a general recursive scheme for the creation of a
clone of an arbitrary frame:

send {\it copy\/} message to each element in the list of descendants;

generate copy of the original frame descriptor;

attach copies of descendants to the copy of descriptor

\noindent The essential point here is the use of new {\it outgoing\/} messages in order
to process a given {\it incoming\/} message. We can regard message processing
as a transformation that maps incoming messages into a set of outgoing
messages, with possible side-effects. The simplest case of such a
transformation is known as {\it delegation\/}. In this case, the input message
is simply passed on to the descendant(s).

As a fine point we clarify that the above algorithm is designed to
create a {\it deep\/} copy of a composite object (a menu viewer in our
case). If a {\it shallow\/} copy would be desired, the descendants would not
have to be copied, and the original descendants instead of their
copies would be attached to the copy of the composite object.

Another example of message handling is provided by mouse
tracking. Assume that a {\it mouse message\/} is received by a menu viewer
while the mouse is located in the upper part of its menu frame and the
left mouse key is kept down. This means ``change viewer's height by
moving its top line vertically''. No message to express the required
transformation of the sub-frames yet exists. Consequently, module
{\tt MenuViewers\/} takes advantage of our open (extensible) message model and
simply introduces an appropriate message type called {\tt ModifyMsg\/}:
\begintt
ModifyMsg = RECORD (Display.FrameMsg)
  id: INTEGER;
  dY, Y, H: INTEGER
END;
\endtt

\noindent The field {\tt id\/} specifies one of two variants: {\tt extend\/} or {\tt reduce\/}. The
first variant of the message requests the receiving frame to move by
the vertical translation vector {\tt dY\/} and then to extend to height {\tt H\/} at
bottom. The second variant requests the frame to reduce to height {\tt H\/} at
bottom and then to move by {\tt dY\/}. In both cases {\tt Y\/} indicates the
Y-coordinate of the new lower-left corner. Figure 4.7 summarizes this
graphically.

Messages arriving from the viewer manager and requesting the receiving
viewer to extend or reduce at its bottom are also mapped into messages
of type {\tt ModifyMsg\/}. Of course, no translation is needed in these cases,
and {\tt dY\/} is 0.

The attentive reader might perhaps have asked why the standard handler
is exported by module {\tt MenuViewers\/} at all. The thought behind is
reusability of code. For example, a message handler for a subclass of
menu viewers could be implemented effectively by reusing menu viewer's
standard handler. After having handled all new or differing cases
first it would simply (super-)call the standard handler subsequently.

TODO: Figure 4.7 The modify frame operation

\beginsubsubsection 4.4.3. Cursor Management.

Traditionally, a cursor indicates and visualizes on the screen the
current location of the {\it caret\/} in a text or, more generally, the
current {\it focus\/} of attention. A small arrow or similar graphic symbol is
typically used for this purpose. In Oberon, we have slightly
generalized and abstracted this concept. A cursor is a path in the
logical display area whose current position can be made visible by a
{\it marker\/}.

The viewer manager and the cursor handler are two concurrent users of
the same display area. Actually, we should imagine two parallel
planes, one displaying viewers and the other displaying cursors. If
there is just one physical plane we take care of painting markers
non-destructively, for example in inverse-video mode. Then, no
precondition must be established before drawing a marker. However, in
the case of a viewer task painting destructively in its viewer's area,
the area must be locked first after turning invisible all markers in
the area.

The technical support of cursor management is again contained in
module {\tt Oberon\/}. The corresponding application programming interface is
\begintt
DEFINITION Oberon;
  TYPE Marker = RECORD
      Fade, Draw: PROCEDURE (x, y: INTEGER)
    END;
    Cursor = RECORD
      marker: Marker;
      on: BOOLEAN;
      X, Y: INTEGER
    END;
  VAR Arrow, Star: Marker;
      Mouse, Pointer: Cursor;
  PROCEDURE OpenCursor (VAR c: Cursor);
  PROCEDURE FadeCursor (VAR c: Cursor);
  PROCEDURE DrawCursor (VAR c: Cursor; VAR m: Marker; X, Y: INTEGER);
  PROCEDURE MarkedViewer (): Viewers.Viewer;
  PROCEDURE RemoveMarks (X, Y, W, H: INTEGER);
  ...
END Oberon.
\endtt

\noindent The state of a cursor is given by its mode of visibility (on), its
position (X, Y) in the display area, and the current marker. {\tt Marker\/} is
an abstract data type with an interface consisting of two operations
{\tt Fade\/} and {\tt Draw\/}. The main benefit we can draw from this abstraction is
once more conceptual independence of the underlying hardware. For
example, {\tt Fade\/} and {\tt Draw\/} can adapt to a given monitor hardware with
built-in cursor support or, in case of absence of such support, can
simply be implemented as identical procedures (an involution) drawing
the marker pattern in inverse video mode.

The functional interface to cursors consists of three operations:
{\tt OpenCursor\/} to open a new cursor, {\tt FadeCursor\/} to switch off the marker
of an open cursor, and {\tt DrawCursor\/} to extend the path of a cursor to a
new position and mark it with the given marker. We emphasize that the
marker representing a given cursor can change its shape dynamically on
the fly.

Two cursors, {\tt Mouse\/} and {\tt Pointer\/} are predefined. They represent the
mouse and an interactively controlled global system pointer
respectively. Typically (but not necessarily) these cursors are
visualized by the built-in markers {\tt Arrow\/} (a small arrow pointing to
north-west) and {\tt Star\/} (a star symbol) respectively. The pointer can be
used to mark any displayed object. It serves primarily as an implicit
parameter of commands.

Two assisting service procedures {\tt MarkedViewer\/} and {\tt RemoveMarks\/} are
added in connection with the predefined cursors. {\tt MarkedViewer\/} returns
the viewer that is currently marked by the pointer. Its resulting
value is equivalent to {\tt Viewers.This(Pointer.X, Pointer.Y)\/}. {\tt RemoveMarks\/}
turns invisible the predefined cursors within a given rectangle in the
display area. This procedure is used to lock the rectangle for its
caller.

Summary of the essential points and characteristics of Oberon's concept of cursor handling:

1.) By virtue of the use of abstract markers and of the logical display area, any potential hardware dependence is encapsulated in system modules and is therefore hidden from the application programmer. Cursors are moving uniformly within the whole display area, even across screen boundaries.

2.) Cursor handling is decentralized by delegating it to the individual handlers that are installed in viewers. Typically, a handler reacts on the receipt of a mouse tracking message by drawing the mouse cursor at the indicated new position. The benefit of such individualized handling is flexibility. For example, a smart local handler might choose the shape of the visualizing marker depending on the exact location, or it might force the cursor onto a grid point.

3.) Even though cursor handling is decentralized, there is some intrinsic support for cursor drawing built into the declaration of type {\tt Cursor\/}. Cursors are objects of full value and, as such, can ``memorize'' their current state. Consequently, the interface operations {\tt FadeCursor\/} and {\tt DrawCursor\/} need to refer to the desired future state only.

4.) Looking at the viewer manager as one user of the display area, the cursor handler is a second (and logically concurrent) user of the same resource. If there is just one physical plane implementing the display area, any region must be locked by a current user before destructive painting. Therefore, markers are usually painted non-destructively in inverse-video mode.

Let us now recapitulate the entire Section. The central resource
managed by the display subsystem is the logical display area whose
purpose is abstraction from the underlying display monitor
hardware. The display area is primarily used by the viewer manager for
the accommodation of tracks and viewers, which are merely the first
two levels of a potentially unlimited nested hierarchy of display
frames. For example, standard menu viewers contain two subordinate
frames: A menu frame and a main frame of contents. Viewers are treated
as black boxes by the viewer manager and are addressed via
messages. Viewers and, more generally frames, are used as elements of
message-based interfaces connecting the display subsystem with other
subsystems like the task scheduler and the various document
managers. Finally, the display area is also the living room of
cursors. In Oberon, a cursor is a marked path. Two standard cursors
Mouse and Pointer are predefined.

\beginsubsection 4.5. RASTER OPERATIONS.

In Section 4.4 we introduced the display area as an abstract concept,
modeled as a two-dimensional Cartesian plane. So far, this view of
the display space was sufficient because we were interested in its
global structure only and ignored contents completely. However, if we
are interested in the displayed contents, we need to reveal more
details about the model.

The Cartesian plane representing the display area is discrete. We
consider points in the display area as grid points or picture elements
(pixels), and we assume contents to be generated by assigning colors
to the pixels. For the moment, the number of possible colors a pixel
can attain is irrelevant. In the binary case of two colors we think of
one color representing background and the other color representing
foreground.

The most elementary operation generating contents in a discrete plane
is ``set color of pixel'' or ``set pixel'' for short. While a few drawing
algorithms directly build on this atomic operation, block-oriented
functionality (traditionally called raster operations) plays a much
more important role in practice. By a block we mean a rectangular area
of pixels whose bounding lines are parallel to the axes of the
coordinate system.

Raster operations are based on a common principle: A block of width {\tt SW\/}
and height {\tt SH\/} of source pixels is placed at a given point of
destination {\tt (DX, DY)\/} in the display area. In the simplest case, the
destination block {\tt (DX, DY, SW, SH)\/} is plainly overwritten by the
source block. In general, the new value of a pixel in the destination
block is a combination of its old value and the value of the
corresponding source pixel:

$d := F(s, d)$

{\it F\/} is sometimes called the {\it mode\/} of combination of the
raster operation. The raster is stored as an array of values of type
{\tt SET\/}, each set representing 32 black/white pixels. The modes of
combining source and destination is implemented by the following set
operations:

TODO: table
mode
replace paint invert
operation
s
s + d (or) s / d (xor)

Note that invert is equivalent with inverse video mode if {\tt s\/} is {\tt TRUE\/}
for all pixels.

There are many different variants of raster operations. Some refer to
a source block in the display area, others specify a constant pattern
to be taken as source block. Some variants require replication of the
source block within a given destination block {\tt (DX, DY, DW, DH)\/} rather
than simple placement.

The challenge when designing a raster interface is finding a unified,
small and complete set of raster operations that covers all needs, in
particular including the need of placing character glyphs. The
amazingly compact resulting set of Oberon raster operations is
exported by module {\tt Display\/}:
\begintt
DEFINITION Display;
  CONST black = 0; white = 1; (*colors*)
        replace = 0; paint = 1; invert = 2; (*operation modes*)
  PROCEDURE Dot (col, x, y, mode: INTEGER);
  PROCEDURE ReplConst (col, x, y, w, h, mode: INTEGER);
  PROCEDURE CopyPattern (col, patadr, x, y, mode: INTEGER);
  PROCEDURE CopyBlock (sx, sy, w, h, dx, dy, mode: INTEGER);
  PROCEDURE ReplPattern (col, patadr, x, y, w, h, mode: INTEGER);
END Display.
\endtt

\noindent In the parameter lists of the above raster operations, {\tt mode\/} is the
mode of combination (replace, paint, or invert). {\tt  CopyBlock\/} copies the
source block {\tt (sx, sy, w, h)\/} to position {\tt (dx, dy)\/} and uses mode to
combine new contents in the destination block {\tt (dx, dy, w, h)\/}. It is
assumed tacitly that the numbers of colors per pixel in the source
block and in the destination area are identical. It is perhaps
informative to know that {\tt CopyBlock\/} is essentially equivalent with the
famous {\tt BitBlt\/} (bit block transfer) in the SmallTalk project
\footnote1{[Goldberg] A. Goldberg, Smalltalk-80: The Interactive Programming Environment, Addison-Wesley 1984.}. In Oberon, {\tt CopyBlock\/} is used primarily for scrolling
contents within a viewer.

The remaining raster operations use a constant pattern. Patterns are
implemented as arrays of bytes, and the parameter patadr is the
address of the relevant pattern. The first two bytes indicate width {\tt w\/}
and height {\tt h\/} of the pattern. Pattern data are given as a sequence of
bytes to be placed into the destination block from left to right and
from bottom to top. Each line takes an integral number of
bytes. Hence, the number of data bytes is $((w+7) \div 8) \times h$. An
example is shown in Figure 4.8.

TODO: Figure 4.8 A pattern and its encoding as an array of bytes (in hex)

Some standard patterns are included in module {\tt Display\/} and exported as
global variables. Among them are patterns arrow, hook, and star
intended to represent the cursor, the caret, and the marker. A second
group of predefined patterns supports drawing graphics.

The parameter col in the pattern-oriented raster operations specifies
the pattern's foreground color. Colors black (background) and white
are predefined. Procedure {\tt CopyPattern\/} copies the pattern to location
x, y in the display area, using the given combination mode. It is
probably the most frequently used operation of all because it is
needed to write text. Procedure {\tt ReplPattern\/} replicates the given
pattern to the given destination block. It starts at bottom left and
proceeds from left to right and from bottom to top. Procedures {\tt Dot\/} and
{\tt ReplConst\/} are special cases of {\tt CopyPattern\/} and {\tt ReplPattern\/}
respectively, taking a fixed implicit pattern consisting of a single
foreground pixel. {\tt Dot\/} is exactly our previously mentioned ``set pixel''.
{\tt ReplConst\/} is used to draw horizontal and vertical lines of
various widths.

The raster operations are a prominent example of the use of Oberon's
data type {\tt SET\/}. Formally, variables are sets of integers between 0 and
31. Here, they are taken as sets of bits numbered from 0 to 31. We
consider the replication of 1's (mode = replace or paint) in the
rectangle with origin x, y, width w, and height h. Every line consists
of 1024 pixels, or 32 words. al, ar, a0, a1 are addresses.
\begintt
VAR al, ar, a0, a1: INTEGER;
    left, right, pixl, pixr: SET;
al := base + y*128;
ar := ((x+w-1) DIV 32)*4 + al; al := (x DIV 32)*4 + al;
left := {(x MOD 32) .. 31}; right := {0 .. ((x+w-1) MOD 32)};
FOR a0 := al TO al + (h-1)*128 BY 128 DO
  SYSTEM.GET(a0, pixl); SYSTEM.GET(ar, pixr);
  SYSTEM.PUT(a0, pixl + left);
  FOR a1 := a0+4 TO ar-4 BY 4 DO SYSTEM.PUT(a1, {0 .. 31}) END;
  SYSTEM.PUT(ar, pixr + right)
END
\endtt

\noindent The definition (and even more so the implementation) of module {\tt Display\/}
provides support for a restricted class of possible hardware
configurations only. Any number of display monitors is theoretically
possible. However, they must be mapped to a regular horizontal array
of predefined cells in the display area. Each cell is vertically split
into two congruent regions, where the corresponding monitor is
supposed to be able to select and display one of the two regions
alternatively. Finally, it is assumed that all cells hosting
black-and-white monitors are allocated to the left of all cells
hosting color monitors. Figure 4.9 gives an impression of such a
configuration.

TODO: Figure 4.9 General, regular cell structure of display area

Under these restrictions any concrete configuration can be
parameterized by the variables of the definition above. {\tt Unit\/}, {\tt Width\/},
and {\tt Height\/} specify the extent of a displayed region, where {\tt Width\/} and
{\tt Height\/} are width and height in pixel units, and {\tt Unit\/} is the size of a
pixel in units of 1/36’000 mm. 1/36’000 mm is a common divisor of all
of the standard metric units used by the typesetting community, like
mm, inch, Pica point and point size of usual printing devices. {\tt Bottom\/}
and {\tt UBottom\/} specify the bottom y-coordinate of the primary region and
the secondary region respectively. Finally, {\tt Left\/} and {\tt ColLeft\/} give the
left x-coordinate of the area of black-and-white monitors and of color
monitors respectively.

\beginsubsection 4.6. STANDARD DISPLAY CONFIGURATIONS AND TOOLBOX.

Let us now take up again our earlier topic of configuring the display
area. We have seen that no specific layout of the display area is
distinguished by the general viewer management itself. However, some
support of the familiar standard Oberon display look is provided by
module {\tt Oberon\/}.

In the terminology of this module, a standard configuration consists
of one or several horizontally adjacent displays, where a display is a
pair consisting of two tracks of equal height, a user track on the
left and a system track on the right. Note that even though no
reference to any physical monitor is made, a display is typically
associated with a monitor in reality.

This is the relevant excerpt of the definition:
\begintt
DEFINITION Oberon;
  PROCEDURE OpenDisplay (UW, SW, H: INTEGER);
  PROCEDURE OpenTrack (X, W: INTEGER);
  PROCEDURE DisplayWidth (X: INTEGER): INTEGER;
  PROCEDURE DisplayHeight (X: INTEGER): INTEGER;
  PROCEDURE UserTrack (X: INTEGER): INTEGER;
  PROCEDURE SystemTrack (X: INTEGER): INTEGER;
  PROCEDURE AllocateUserViewer (DX: INTEGER; VAR X, Y: INTEGER);
  PROCEDURE AllocateSystemViewer (DX: INTEGER; VAR X, Y: INTEGER);
END Oberon.
\endtt

\noindent Procedure {\tt OpenDisplay\/} initializes and opens a new display of the
dimensions {\tt H\/} (height), {\tt UW\/} (width of user track), and {\tt SW\/} (width of
system track). Procedure {\tt OpenTrack\/} overlays the sequence of existing
tracks spanned by the segment $[X, X + W)$ by a new track. Both
procedure {\tt OpenDisplay\/} and {\tt OpenTrack\/} take from the client the burden
of creating a filler viewer.

The next group of procedures {\tt DisplayWidth\/}, {\tt DisplayHeight\/},
{\tt UserTrack\/} and {\tt SystemTrack\/} return
width or height of the respective structural entity located at
position X in the display area.

Procedures {\tt AllocateUserViewer\/} and {\tt AllocateSystemViewer\/} make proposals
for the allocation of a new viewer in the desired track of the display
located at {\tt DX\/}. In first priority, the location is determined by the
system pointer that can be set manually. If the pointer is not set, a
location is calculated on the basis of some heuristics whose
strategies rely on different splitting fractions that are applied in
the user track and in the system track respectively, with the aim of
generating aesthetically satisfactory layouts.

In addition to the programming interface provided by module {\tt Oberon\/} for
the case of standard display layouts, the display management section
in the {\tt System\/} toolbox provides a user interface:
\begintt
DEFINITION System; (*Display management*)
  PROCEDURE Open; (*viewer*)
  PROCEDURE Close; (*viewer*)
  PROCEDURE CloseTrack;
  PROCEDURE Recall; (*most recently closed viewer*)
  PROCEDURE Copy; (*viewer*)
  PROCEDURE Grow; (*viewer*)
  PROCEDURE Clear; (*clear system log*)
END System.
\endtt

\noindent In turn, these commands are called to open a text viewer in
the system track, close a viewer, close a track, recall (and reopen)
the most recently closed viewer, copy a viewer, and grow a viewer. The
commands {\tt Close\/}, {\tt CloseTrack\/}, {\tt Recall\/}, {\tt Copy\/},
and {\tt Grow\/} are generic. {\tt Close\/}, {\tt Copy\/},
and {\tt Grow\/} are typically included in the title bar of a menu
viewer. Their detailed implementations follow subsequently.

TODO: References


\beginchapter CHAPTER 5: THE TEXT SYSTEM.


% \beginsubsection 5.1. TEXT AS AN ABSTRACT DATA TYPE.
% \beginsubsubsection 5.1.1. Loading and Storing Text.
% \beginsubsubsection 5.1.2. Editing Text.
% \beginsubsubsection 5.1.3. Accessing Text.
% \beginsubsection 5.2. TEXT MANAGEMENT.
% \beginsubsection 5.3. TEXT FRAMES.
% \beginsubsection 5.4. THE FONT MACHINERY.
% \beginsubsection 5.5. THE EDIT TOOLBOX.
% \beginchapter CHAPTER 6: THE MODULE LOADER.
% \beginsubsection 6.1. LINKING AND LOADING.
% \beginsubsection 6.2. MODULE REPRESENTATION.
% \beginsubsection 6.3. THE LINKING LOADER.
% \beginsubsection 6.4. THE TOOLBOX OF THE LOADER.
% \beginsubsection 6.5. THE OBERON OBJECT FILE FORMAT.
% \beginchapter CHAPTER 7: THE FILE SYSTEM.
% \beginsubsection 7.1. FILES.
% \beginsubsection 7.2 IMPLEMENTATION OF FILES ON A RANDOM-ACCESS STORE.
% \beginsubsection 7.3 IMPLEMENTATION OF FILES ON A DISK.
% \beginsubsection 7.4 THE FILE DIRECTORY.
% \beginsubsection 7.5. THE TOOLBOX OF FILE UTILITIES.
% \beginchapter CHAPTER 8: STORAGE LAYOUT AND MANAGEMENT.
% \beginsubsection 8.1. LAYOUT AND RUN-TIME ORGANIZATION.
% \beginsubsection 8.2. MANAGEMENT OF DYNAMIC STORAGE.
% \beginsubsection 8.3. THE KERNEL.
% \beginsubsection 8.4. THE STORAGE MANAGEMENT'S TOOLBOX.
% \beginchapter CHAPTER 9: DEVICE DRIVERS.
% \beginsubsection 9.1. OVERVIEW.
% \beginsubsection 9.2. KEYBOARD AND MOUSE.
% \beginsubsection 9.3. THE SD-CARD (DISK).
% \beginsubsection 9.4. SERIAL ASYNCHRONOUS INTERFACE (RS 232).
% \beginsubsection 9.5. SERIAL COMMUNICATIONS CONTROLLER (SCC).


\bye
